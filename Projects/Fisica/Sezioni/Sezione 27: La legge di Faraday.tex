\documentclass[../Main/Appunti Fisica.tex]{subfiles}
\begin{document}
Si consideri una spira di filo conduttore connessa ad un amperometro ad altissima sensibilità.
Se alla spira viene avvicinato un magnete, l'amperometro indicherà un valore negativo, viceversa se lo si allontana sarà indicato un valore positivo.
\\ \\
Se ne deduce che vi è una relazione tra corrente e variazione del campo magnetico.
\\ \\
Da alcune osservazioni fatte, Faraday dedusse che una corrente può essere indotta da un campo magnetico variabile.\\
Il che conduce alla \textit{legge di Faraday}, la quale stabilisce
\[
    \varepsilon_{ind} = \dv{\Phi_{B}}{t}
\]
cioè la \fem indotta è conseguenza di una variazione del flusso magnetico.

\subsection{F.e.m. nei circuiti in moto.}
\subfile{../Sezioni/Sotto Sezioni/Sezione 27: SottoSezione 1.tex}

\subsection{Legge di Lenz.}
\subfile{../Sezioni/Sotto Sezioni/Sezione 27: SottoSezione 2.tex}

\subsection{F.e.m. indotte e campi elettrici indotti.}
\subfile{../Sezioni/Sotto Sezioni/Sezione 27: SottoSezione 3.tex}
\end{document}