\documentclass[../Main/Appunti Fisica.tex]{subfiles}
\begin{document}
Il concetto di energia, è utile poiché applicabile ai sistemi meccanici, senza necessariamente ricorrere alle leggi di Newton.
Inoltre, l'approccio energetico è utile nella comprensione dei fenomeni elettrici e termici, per i quali non è possibile applicare Newton.

\subsection{Sistema e ambiente esterno.}
\subfile{../Sezioni/Sotto Sezioni/Sezione 10: SottoSezione 1.tex}

\subsection{Lavoro compiuto da una forza costante.}
\subfile{../Sezioni/Sotto Sezioni/Sezione 10: SottoSezione 2.tex}
\clearpage

\subsection{Lavoro compiuto da una forza variabile.}
\subfile{../Sezioni/Sotto Sezioni/Sezione 10: SottoSezione 3.tex}
\clearpage

\subsection{Energia cinetica e teorema dell'energia cinetica.}
\subfile{../Sezioni/Sotto Sezioni/Sezione 10: SottoSezione 4.tex}
\clearpage

\subsection{Energia potenziale di un sistema.}
\subfile{../Sezioni/Sotto Sezioni/Sezione 10: SottoSezione 5.tex}
\clearpage

\subsection{Forze conservative e non conservative.}
\subfile{../Sezioni/Sotto Sezioni/Sezione 10: SottoSezione 6.tex}

\subsection{Forze conservative ed energia potenziale.}
\subfile{../Sezioni/Sotto Sezioni/Sezione 10: SottoSezione 7.tex}
\end{document}