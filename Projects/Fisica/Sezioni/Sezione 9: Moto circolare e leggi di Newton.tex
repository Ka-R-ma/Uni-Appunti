\documentclass[../Main/Appunti Fisica.tex]{subfiles}
\begin{document}
Si consideri un corpo collegato ad una fune, tenuto in moto circolatorio da un velocità costante in modulo, su un piano privo di attrito.
Dalla seconda legge di Newton, se sul corpo non sono applicate forze, questi si muove secondo una traiettoria rettilinea;
ma la corda esercita sul corpo un forza radiale \(\va{F_{r}}\), che obbliga il moto circolatorio.
\\ \\
Pertanto, applicando la seconda legge, segue
\[
    \sum \vb{F} = \vb{F_{r}} = m\vb{a_{c}} = m\frac{\vb{v^{2}}}{r}
\]
ove \(\vb{a_{c}}\) è definita accelerazione centripeta.

\subsection{Moto in sistemi di riferimento accelerati.}
\subfile{../Sezioni/Sotto Sezioni/Sezione 9: Sottosezione 1.tex}

\subsection{Moto in presenza di forze frenanti.}
\subfile{../Sezioni/Sotto Sezioni/Sezione 9: SottoSezione 2.tex}
\end{document}