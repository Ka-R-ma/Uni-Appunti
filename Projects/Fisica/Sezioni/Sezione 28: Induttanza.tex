\documentclass[../Main/Appunti Fisica.tex]{subfiles}
\begin{document}
Si consideri la figura di seguito riportata.
\begin{figure}[!h]
    \centering
    \begin{tikzpicture}[scale = 1, every node/.style={scale=1}]

        \node (source) [battery1shape, rotate = -90] at (0, 0) {};
        \node [anchor = east] at (-0.25, 0) {\(\varepsilon\)};

        \draw (source.left) to ($(source.left) + (0, 0.75)$) to ($(source.left) + (2, 0.75)$) to [ccgsw] ($(source.right) + (2, -0.75)$) to [R]
        ($(source.right) + (0, -0.75)$) to (source.right);

        \draw [-stealth, thick] (1.5, -1.5) -- (0.5, -1.5);
        \node (I) [anchor = north] at (1, -1.5) {\(I\)};

    \end{tikzpicture}
    \caption{Circuito soggetto ad induzione.}
    \label{fig:20}
\end{figure}

Quando l'interruttore è chiuso la corrente non passa istantaneamente da 0 a \(\varepsilon / R\).
Infatti se si analizza applicando Faraday, non appena la corrente aumenta si ha una variazione del flusso magnetico.
Tale \fem produce una corrente indotta opposta alla corrente fornita dalla batteria.
\\ \\
Poiché tale fenomeno è generato dal circuito stesso, si definisce \textit{autoinduzione}.
\\ \\
Da Faraday segue dunque
\begin{equation}\label{eq:19}
    \varepsilon_{r} = L \dv{I}{t}
\end{equation}
ove \(L\) è una costante di proporzionalità, detta \textit{induttanza}, dipendente dalle caratteristiche fisiche e geometriche della spira stessa.
\\ \\
Si consideri una spira composta da \(N\) spire, segue
\[
    L = \frac{N \Phi_{B}}{I}
\]
posto \(\Phi_{B}\) lo stesso per ogni spira ed \(L\) induttanza dell'intera spira.
\\ \\
Dall'equazione \eqref{eq:19}
\[
    L = - \frac{\varepsilon}{\dd \Phi_{B} / \dd t}
\]

Nel SI, l'induttanza è misurata in Henry, (\SI[unit-color = MidnightBlue]{}{\henry}).
\begin{center}
    \SI[unit-color = MidnightBlue, per-mode = fraction]{1}{\henry \equiv \volt \cdot \second \per \ampere}
\end{center}
\clearpage

\subsection{Circuiti RL.}
\subfile{../Sezioni/Sotto Sezioni/Sezione 28: SottoSezione 1.tex}

\subsection{Energia di un campo magnetico.}
\subfile{../Sezioni/Sotto Sezioni/Sezione 28: SottoSezione 2.tex}
\end{document}