\documentclass[../Main/Appunti Fisica.tex]{subfiles}
\begin{document}
Si definisce un sistema composto da due cariche, \textit{condensatore}.
\\ \\
Da osservazioni sperimentali \(Q \propto \Delta V\), cioè \(Q = C \Delta V\) ove \(C\) è detta capacità.

\begin{Definition*}
    La capacità \(C\) di un condensatore è data dal rapporto tra il valore assoluto della carica e la differenza di potenziale tra i conduttori.
    \begin{equation}\label{eq:17}
        C \equiv \frac{\abs{Q}}{\Delta V}
    \end{equation}
\end{Definition*}
Nel SI, la capacità è misurata in Farad, (\SI[unit-color = MidnightBlue]{}{\farad}).
\begin{center}
    \SI[unit-color = MidnightBlue, per-mode = fraction]{1}{\farad \equiv \coulomb \per \volt}
\end{center}

\subsection{Calcolo della capacità di un condensatore piano.}
\subfile{../Sezioni/Sotto Sezioni/Sezione 22: SottoSezione 1.tex}

\subsection{Combinazioni di condensatori.}
\subfile{../Sezioni/Sotto Sezioni/Sezione 22: SottoSezione 2.tex}

\subsection{Energia immagazzinata in un condensatore carico.}
\subfile{../Sezioni/Sotto Sezioni/Sezione 22: SottoSezione 3.tex}
\end{document}