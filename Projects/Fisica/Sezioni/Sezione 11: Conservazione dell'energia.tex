\documentclass[../Main/Appunti Fisica.tex]{subfiles}
\begin{document}
Si è fin'ora parlato di un unico metodo per cui l'energia può essere trasferita da un sistema ad un'alto, ossia il lavoro.
Esistono però altri metodi di trasferimento di energia quali
\begin{itemize}
    \item \underline{onde meccaniche}: processi nei quali l'energia è trasportata da perturbazioni che si propagano nell'aria;
    \item \underline{calore}: meccanismo di trasferimento dovuto a una differenza di temperatura tra du regioni dello spazio.
\end{itemize}
%
Uno degli aspetti fondamentali di questa sezione è il seguente fatto, l'energia non può essere ne creata ne distrutta, questa si conserva sempre.

\begin{Principle*}[di conservazione dell'energia]
    Se l'energia presente in un sistema subisce una variazione, è necessario che una pari quantità di energia abbia lasciato il sistema.
    \[
        \Delta E_{s} = \sum T
    \]
\end{Principle*}
ove \(E_{s}\) indica l'energia del sistema, \(T\) la quantità di energia trasferita tramite i meccanismi prima citati.

\subsection{Sistema isolato.}
\subfile{../Sezioni/Sotto Sezioni/Sezione 11: SottoSezione 1.tex}

\subsection{Sistemi con attrito dinamico.}
\subfile{../Sezioni/Sotto Sezioni/Sezione 11: SottoSezione 2.tex}
\clearpage

\subsection{Forze non conservative e variazione dell'energia meccanica.}
\subfile{../Sezioni/Sotto Sezioni/Sezione 11: SottoSezione 3.tex}

\subsection{Potenza.}
\subfile{../Sezioni/Sotto Sezioni/Sezione 11: SottoSezione 4.tex}
\clearpage
\end{document}