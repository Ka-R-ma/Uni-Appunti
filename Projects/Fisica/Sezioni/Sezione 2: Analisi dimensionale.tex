\documentclass[../Main/Appunti Fisica.tex]{subfiles}
\begin{document}
In fisica il concetto di dimensionale ha un significato particolare, con esso infatti si caratterizza la natura di una grandezza.
\\ \\
Si avrà spesso il bisogno di verificare la correttezza di un'equazione, in questi casi uno strumento utile è l'analisi dimensionale.
Questa permette di verificare se tutti i membri di una certa equazione appartengano alla stessa dimensione.

\begin{Example*}
    \[
        x(t) = x_{0} + v_{0}t + \frac{1}{2}at^{2}
    \]
    L'equazione sopra indicata è dimensionalmente consistente poiché, se analizzata dal punto di vista delle unità di misura si ha
    \[
        \abs{L} = \abs{L} + \frac{\abs{L}}{\abs{T}}\abs{T} + \frac{\abs{L}}{\abs{T}^{2}}\abs{T}^{2}
    \]
    Viceversa un'equazione del tipo
    \[
        x(t) = x_{0} + v_{0}t + \frac{1}{2}at^{3}
    \]
    risulta essere inconsistente dimensionalmente poiché, se analizzata dal punto di vista delle unità di misura si ha
    \[
        \abs{L} = \abs{L} + \frac{\abs{L}}{\abs{T}}\abs{T} + \frac{\abs{L}}{\abs{T}^{2}}\abs{T}^{3}
    \]
    il che è inconcepibile.
\end{Example*}
\end{document}
\clearpage