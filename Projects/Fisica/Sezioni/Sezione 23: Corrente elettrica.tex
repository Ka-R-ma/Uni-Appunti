\documentclass[../Main/Appunti Fisica.tex]{subfiles}
\begin{document}
Si faccia l'ipotesi di alcune cariche che si muovono perpendicolarmente ad una superficie di area \(A\).
Se si definisce \(\Delta Q\) la carica che attraversa la superficie in un tempo \(\Delta t\), allora
\[
    \mv{I} = \frac{\Delta Q}{\Delta t}
\]
da cui
\[
    I = \frac{\dd Q}{\dd t}
\]

Nel SI, la corrente è misurata in Ampere, (\SI[unit-color = MidnightBlue]{}{\ampere}).
\begin{center}
    \SI[unit-color = MidnightBlue, per-mode = fraction]{1}{\ampere \equiv \coulomb \per \second}
\end{center}

\subsubsection{Modello microscopico della corrente.}
Si consideri un elemento conduttore di lunghezza \(\Delta x\), il cui volume sia \(\Delta x A\). Siano \(n\) il numero di portatori di carica per unita di volume,
allora
\[
    \Delta Q = (n A \Delta x) q
\]
ove \(q\) è la carica di ciascun portatore di carica.\\
Se se ciascun portatore si muove a velocità \(\vb{v_{d}}\), poiché \(\Delta x = \vb{v_{d}} \Delta t\), segue
\[
    \Delta Q = (n A \vb{v_{d}} \Delta t) q
\]
da cui segue
\[
    \mv{I} = \frac{\Delta Q}{\Delta t} = n A \vb{v_{d}} q
\]
\clearpage

\subsection{Resistenza.}
\subfile{../Sezioni/Sotto Sezioni/Sezione 23: SottoSezione 1.tex}

\subsection{Modello di conduzione elettrica.}
\subfile{../Sezioni/Sotto Sezioni/Sezione 23: SottoSezione 2.tex}

\subsection{Resistenza e temperatura.}
\subfile{../Sezioni/Sotto Sezioni/Sezione 23: SottoSezione 3.tex}

\subsection{Potenza elettrica.}
\subfile{../Sezioni/Sotto Sezioni/Sezione 23: SottoSezione 4.tex}
\end{document}