\documentclass[../Main/Appunti Fisica.tex]{subfiles}
\begin{document}
La misurazione di grandezze fisiche comporta spesso una correttezza dei valori ottenuti solo entro certi limiti.
Tale incertezza è dovuta a vari fattori, quale ad esempio la qualità dello strumento utilizzato.
\\ \\
Al fine di dare una corretta rappresentazione del valore, si usano spesso le cifre significative.
(Nei presenti appunti ci si limiterà alla terza cifra.)\\
Quando si svolgono operazioni, per assegnare un corretto numero di cifre significative si utilizzano le seguenti regole.
\begin{enumerate}
    \item \underline{Operazioni di somma e sottrazione}: il numero di cifre significative del risultato deve essere uguale al numero di cifre significative degli operandi.
          \begin{Example*}
              \[
                  12.3 + 1.92 = 14.2 \text{\color{red} non } 14.22
              \]
          \end{Example*}

    \item \underline{Operazioni di prodotto e divisione}: il numero di cifre significative deve essere uguale al numero minimo di cifre significative degli operandi.
          \begin{Example*}
              \[
                  12.1 \cdot 1.001 = 12.1 \text{\color{red} non } 12.10121
              \]
          \end{Example*}
\end{enumerate}
\end{document}
\clearpage