\documentclass[../../Main/Appunti Fisica.tex]{subfiles}
\begin{document}
Si supponga di strofinare un palloncino contro la manica di un maglione.
Se al palloncino si avvicinano dei pezzetti di carta, questi saranno attratti dallo stesso, dimostrando l'esistenza di una forza elettrica.
\\ \\
Dall'esempio fatto si evince l'esistenza di due tipi di cariche elettriche, che per semplicità si definiscono positiva e negativa.
\\ \\
Si supponga ora di avvicinare due palloncini, caricati staticamente come in precedenza.
Si noterà subito che questi tendono a respingersi.
Si può dunque concludere che cariche simili si respingono e cariche opposte si attraggono.
\begin{Note*}
    La carica elettrica si conserva sempre.
\end{Note*}
\end{document}