\documentclass[../,,/Main/Appunti Fisica.tex]{subfiles}
\begin{document}
Si supponga un gas a volume iniziale \(V_{i}\), libero di occupare un volume \(V_{f}\).
Supponendo \(V_{m}\) la massa delle singole particelle, il numero di possibili locazioni \(\omega\) delle molecole è
\[
    \omega_{i} = \frac{V_{i}}{V_{m}}
\]
%
Trascurando la possibilità che due particelle siano nella medesima posizione, \(N\) molecole potranno trovarsi in \(\omega_{i}^{N}\) stati diversi.
Cioè
\[
    \omega_{i}^{N} = \left( \frac{V_{i}}{V_{m}} \right)^{N} = \Omega_{i}
\]
analogamente il numero di micro-stati finali sara dato da
\[
    \omega_{i}^{N} = \left( \frac{V_{f}}{V_{m}} \right)^{N} = \Omega_{f}
\]
%
Calcolando ora il rapporto tra i micro-stati finali e iniziali, segue
\[
    \frac{\Omega_{f}}{\Omega_{i}} = \left( \frac{V_{f}}{V_{i}} \right)^{N}
\]
da cui moltiplicando per \(K_{B}\) è calcolandone il logaritmo, segue
\[
    K_{B} \ln\left( \frac{\Omega_{f}}{\Omega_{i}} \right) = n N_{a} K_{B} \ln \left( \frac{V_{f}}{V_{i}} \right) = n R \ln \left( \frac{V_{f}}{V_{i}} \right)
\]
segue
\[
    S_{f} - S_{i} =  n R \ln \left( \frac{V_{f}}{V_{i}} \right)
\]
da cui si si ha la seguente definizione di entropia.
\[
    S = K_{b} \ln \Omega
\]
\end{document}
\clearpage