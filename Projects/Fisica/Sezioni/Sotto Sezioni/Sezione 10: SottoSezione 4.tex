\documentclass[../../Main/Appunti Fisica.tex]{subfiles}
\begin{document}
Si consideri una forza orizzontale verso destra applicata su un corpo.
Se tale forza causa uno spostamento, il lavoro totale sarà
\[
    \vb{W_{TOT}} = \int{x_{i}}{x_{f}}{\sum \vb{Fx}}{x}
\]
%
Dunque applicando Newton, segue
\[\begin{aligned}
        \vb{W_{TOT}} & = \int{x_{i}}{x_{f}}{\sum \vb{F_{x}}}{x} = \int{x_{i}}{x_{f}}{m\va{a}}{x}                 \\
                     & = \int{x_{i}}{x_{f}}{m \dv{v}{t}}{x} = \int{x_{i}}{x_{f}}{m \dv{v}{x} \dv{x}{t}}{x}       \\
                     & = \int{v_{i}}{v_{f}}{mv}{v}  = \frac{1}{2} m \vb{v_{f}}^{2} - \frac{1}{2} m\vb{v_{i}}^{2}
    \end{aligned}\]
%
ove \(\vb{v_{i}}\) è la velocità del corpo a \(x_{i}\), metre \(\vb{v_{f}}\) quella in \(x_{f}\).
\\ \\
Si definisce la quantità \(\tfrac{1}{2} mv^{2}\), energia cinetica e la si indica con la lettera \(K\).
\[
    K \equiv \frac{1}{2} mv^{2}
\]
%
Da cio segue
\begin{equation}
    \label{eq:1}
    \vb{W} = K_{f} - K_{i} = \Delta K
\end{equation}
%
L'Equazione \eqref{eq:1} è un'importante risultato noto come \textit{teorema dell'energia cinetica}.
\begin{Theorem*}[dell'energia cinetica]
    Quando si compie lavoro su un corpo, e ne consegue una variazione nel modulo della velocità, il lavoro totale è pari alla variazione di energia cinetica.
\end{Theorem*}
\end{document}