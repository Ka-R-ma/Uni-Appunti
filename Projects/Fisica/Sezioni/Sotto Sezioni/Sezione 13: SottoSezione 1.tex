\documentclass[../../Main/Appunti Fisica.tex]{subfiles}
\begin{document}
I fluidi non sono soggetti né a taglio né a trazione, possono solo essere compressi.
\\ \\
Si consideri un corpo immerso in un fluido, chiuso da un pistone leggero.
Si supponga di applicare una forza \(\vb{F}\), è possibile stabilire la pressione del fluido al livello a cui è immerso il corpo,
avendo nota l'area \(A\) del pistone.
Vale quanto segue
\[
    \vb{P} \equiv \frac{\vb{F}}{A}
\]
Nel SI, la pressione è misurata in Pascal, (\SI{}{\pascal})
\begin{center}
    \SI[per-mode = fraction, unit-color = MidnightBlue]{1}{\pascal \equiv \newton \per \metre\tothe{2}}
\end{center}
\end{document}