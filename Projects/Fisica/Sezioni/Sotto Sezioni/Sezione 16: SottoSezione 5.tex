\documentclass[../../Main/Appunti Fisica.tex]{subfiles}
\begin{document}
Il \textit{primo principio della termodinamica} è un caso di conservazione dell'energia.\\
Quest stabilisce che i soli scambi di energia possono avvenire sotto forma di calore o lavoro, quindi
\[
    \Delta E_{int} = Q + \vb{W}
\]
%
Da ciò segue che, se in un sistema una quantità \(\dd Q\) è scambiata come calore ed è compiuto del lavoro \(\dd \vb{W}\), vi sarà una variazione \(\dd E_{int}\).
\[
    {\dd E_{int}} = {\dd Q} + {\dd \vb{W}}
\]
%
Si consideri un sistema isolato, come detto in precedenza l'energia interna sara costante. Più precisamente poiché \(Q = \vb{W} = 0\) segue che \(E_{int} = 0\).
\\ \\
Si consideri un sistema in cui avviene una trasformazione ciclica, in questo caso
    \[
      \Delta E_{int} = 0 \quad \text{e} \quad Q = - \vb{W}
    \]
\end{document}