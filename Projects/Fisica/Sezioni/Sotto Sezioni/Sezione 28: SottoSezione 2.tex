\documentclass[../../Main/Apputni Fisica.tex]{subfiles}
\begin{document}
Si consideri Figura \ref{fig:21}.
Da questi si osserva che la batteria deve fornire una maggiore quantità di energia, rispetto ad un circuito senza induttore.
\\ \\
Considerando dunque la potenza erogata
\[
    \vb{P} = I \varepsilon = I^{2} R + IL \dv{I}{t}
\]

Ponendo \(U\) l'energia immagazzinata nell'induttore, la velocità \(\dd U / \dd t\) con cui questa è immagazzinata è
\[
    \dv{U}{t} = IL \dv{I}{t}
\]
integrando
\[
    U = \int{}{}{}{U} = \int{0}{I}{LI}{I} = L \int{0}{I}{I}{I}
\]
\begin{equation}\label{eq:21}
    U = \frac{1}{2} L I^{2}
\end{equation}

Si consideri ora un solenoide con induttanza \(L = \mu_{0} n^{2} V\).
Sostituendo all'Equazione \eqref{eq:21} il campo magnetico di un solenoide, segue
\[
    U = \frac{1}{2} IL = \frac{1}{2} \mu_{0}n^{2}V (\frac{B}{\mu_{0} n})^{2} = \frac{\vb{B}^{2}}{2\mu_{0}} V
\]
da ciò l'energia per unita di volume \(\mu_{B}\) è
\begin{equation}\label{eq:22}
    \mu_{B} = \frac{U}{V} =  \frac{\vb{B}^{2}}{2 \mu_{0}}
\end{equation}

\begin{Note*}
    Sebbene derivata per il caso di un solenoide, l'Equazione \eqref{eq:22} è valida per ogni regione dello spazio in cui sia presente un campo magnetico.
\end{Note*}
\end{document}