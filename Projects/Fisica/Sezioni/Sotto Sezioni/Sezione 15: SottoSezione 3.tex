\documentclass[../../Main/Appunti Fisica.tex]{subfiles}
\begin{document}
Per i gas, l'Equazione \eqref{eq:8} non è applicabile, poiché questi non hanno un volume proprio, bensì assumono quello del recipiente che li contiene.
Proprio per tale ragione per i gas si utilizza quella una relazione, definita \textit{equazione di stato}, che lega volume, pressione e temperatura.

\begin{Note*}
    Tutti ragionamenti riguardanti i gas, nella presente sezione e in tutte quelle successive, terranno conto di gas perfetti, ossia a bassa densità.
\end{Note*}

Si consideri un gas contenuto in un cilindro, la cui pressione è variabile tramite un pistone.
Supponendo che non vi siano perdite e la massa rimanga costante, dati sperimentali stabiliscono che
\begin{itemize}
    \item a \(T\) costante la pressione è inversamente proporzionale al volume;
    \item a \(P\) costante il volume è direttamente proporzionale alla temperatura;
    \item a \(V\) costante la pressione è direttamente proporzionale alla temperatura.
\end{itemize}
%
In sintesi
\[
    PV = n R T
\]
ove \(n\) è il numero di moli, \(R\) è la costante dei gas perfetti pari a \SI[per-mode = fraction, unit-color = MidnightBlue]{8.314}{\joule \per \mole \cdot \kelvin}.
\\
Spesso risulta però comodo scrivere la precedente equazione come
\[\begin{aligned}
    PV = nRT &= \frac{N}{N_{a}}RT
    &= NK_{B}T
\end{aligned}\]
ove \(K_{B}\) è la \textit{costante di Boltzmann}, pari a \SI[per-mode = fraction, unit-color = MidnightBlue]{1.38 e-23 }{\joule \per \kelvin}.
\end{document}