\documentclass[../../Main/Apputni Fisica.tex]{subfiles}
\begin{document}
Con la prima legge di Newton, si ha chiaro cosa accada ad un corpo se questi non è soggetto a forze.
Pertanto, è naturale chiedersi cosa accada invece se sul corpo agiscono delle forze.
\\ \\
Si supponga di applicare una forze \(\vb{F}\) su un corpo di massa \(m\), questi inizierà a muoversi, subendo pertanto un'accelerazione.
Incrementando l'intensità della forza \(\vb{F}\), aumenterà di conseguenza l'accelerazione del corpo.
\\ \\
Da tali osservazioni, Newton dedusse che la forza applicata su un corpo è proporzionale all'accelerazione dello stesso, deduzione culminata nella seconda legge.
\begin{Law*}[seconda di Newton]
    L'accelerazione di un corpo, se osservato da un sistema inerziale, è proporzionale alla forza applicata sullo stesso,
    e inversamente proporzionale alla propria massa. Cioè
    \[
        \vb{a} \propto \frac{\sum \vb{F}}{m} \implies \sum \vb{F} = m\vb{a}
    \]
\end{Law*}

Nel SI, la forza è misurata in Newton, \(\SI{}{\newton}\).
\begin{center}
    \SI[per-mode = fraction, unit-color = MidnightBlue]{1}{\newton = \kilogram \cdot \metre \per \second \tothe{2}}
\end{center}
\end{document}