\documentclass[../../Main/Appunti Fisica.tex]{subfiles}
\begin{document}
Si consideri un osservatore ed un corpo posti all'interno di un veicolo in movimento.
Si presentano due scenari
\begin{enumerate}
    \item il veicolo è fermo, dunque anche il corpo rimane fermo;
    \item il veicolo inizia a muoversi, il corpo inizialmente fermo inizia a muoversi verso la parte posteriore del veicolo.
\end{enumerate}
%
Sembra dunque che nel secondo caso si violi la seconda legge di Newton, ma così non è, infatti sul corpo agisce una forza che ne determina il moto;
si definisce tale forza \textit{forza apparente}.
\end{document}