\documentclass[../../Main/Appunti Fisica.tex]{subfiles}
\begin{document}
La resistività di un conduttore varia, entro certi limiti, al variare della temperatura, con la seguente relazione.
\[
    \rho = \rho_{0} \left[ 1 + \alpha(T - T_{0}) \right]
\]
ove \(\alpha\) è detto \textit{coefficiente di resistività termico}.\\
Dalla precedente relazione
\[
    \alpha = \frac{\Delta \rho}{\rho_{0} \Delta T}
\]
ma si è detto che la resistenza è proporzionale alla resistività, segue pertanto
\[
    R = R_{0} \left[ 1 + \alpha(T - T_{0}) \right]
\]
\end{document}