\documentclass[../../Main/Appunti Fisica.tex]{subfiles}
\begin{document}
Si consideri una retta orientata, per stabilire la posizione di un punto \(x \neq 0\), è necessario sapere dove questi si trovi rispetto il punto di riferimento.
\\
Si consideri la seguente retta orientata
\begin{figure}[!h]
    \centering
    \begin{tikzpicture}[scale = 1, every node/.style={scale=1}]

        \draw [->, ultra thick] (-1, 0) -- (3, 0);

        \node [anchor = south] at (0, 0) {0};
        \node [anchor = north] at (1.5, 0) {\(x_{i}\)};
        \node [anchor = north] at (2.5, 0) {\(x_{f}\)};

        \node [circ] at (0, 0) {};
        \node [circ] at (1.5, 0) {};
        \node [circ] at (2.5, 0) {};

    \end{tikzpicture}
\end{figure}
ove sono stati fissati i punti \(x_{i} \text{e} x_{f}\).
\\ \\
Si definisce spostamento, la variazione di posizione del punto in un certo lasso di tempo.
Cioè
\[
    \Delta x = x_{f} - x_{i}
\]
Dalla variazione di posizione si può ricavare la velocità media con cui avviene tale spostamento, la quale è data dalla seguente relazione.
\[
    \vb{v} = \frac{\Delta x}{\Delta t}
\]
Spesso però risulta utile conoscere la velocità del corpo in un certo istante di tempo, in questo caso si parlerà di velocità istantanea ed è data dalla seguente relazione.
\[
    v = \lim\limits_{\Delta t \to 0} \frac{\Delta x}{\Delta t} = \dv{x}{t}
\]
\end{document}
\clearpage