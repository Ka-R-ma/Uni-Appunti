\documentclass[../../Main/Appunti Fisica.tex]{subfiles}
\begin{document}
Si consideri un corpo posto ad una certa altezza dal suolo, poiché su questi non si applica una forza, lo stesso non ha energia cinetica bensì ne possiede una con tali capacità.
Si definisce tale energia \textit{energia potenziale}.
\\ \\
Si consideri ora una agente esterno che solleva un corpo da una posizione \(y_{i}\) a \(y_{f}\), ponendo che la forza applica sia pari alla forza di gravita, segue
\[
    \vb{W} = \va{F}_{app} \cdot \Delta \vb{s} = m\vb{g}y_{f} - m\vb{g}v_{i}
\]
Si definisce la quantità \(m\vb{g}y\), energia potenziale gravitazionale e la si indica con \(U_{g}\).
\[
    U_{g} \equiv m\vb{g}v
\]
Dunque il lavoro effettuato dall'agente esterno è
\[
    \vb{W} = \Delta U_{g}
\]

\subsubsection{Energia potenziale elastica.}
Si consideri Figura \ref{fig:6}.
Si è detto che il lavoro della forza elastica è
\[
    \vb{W} = \int{x_{i}}{x_{f}}{-kx}{x} = \frac{1}{2}kx_{f}^{2} - \frac{1}{2}kx_{i}^{2}
\]
ma questi risulta essere uguale al lavoro applicato dall'agente esterno che allunga o comprime la molla.
Pertanto si può si può mettere in correlazione il lavoro dell'agente esterno con l'energia potenziale infatti,
similarmente a quanto fatto con l'energia potenziale gravitazionale si può stabilire un'\textit{energia potenziale elastica} \(U_{s}\).
\[
    U_{s} = \frac{1}{2} kx^{2}
\]
\end{document}