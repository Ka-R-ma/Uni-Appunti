\documentclass[../../Main/Appunti Fisica.tex]{subfiles}
\begin{document}
Un sistema i cui confini non sono attraversati da nessun flusso di energia si dice \textit{isolato}.
\\ \\
Si consideri un corpo sollevato dal suolo, come detto il lavoro sarà \(\vb{W} = -\Delta U_{g}\).
Se ci si concentra sul lavoro fatto da \(\vb{F_{g}}\), quando il corpo è lasciato cadere al suolo, questi risulta essere
\[\begin{aligned}
        \vb{W_{c}} & = - m\vb{g} \Delta \vb{s}       \\
                   & = - m\vb{g}y_{f} + m\vb{g}y_{i} \\
    \end{aligned}\]
%
Dal teorema dell'energia cinetica
\[
    \vb{W_{c}} = \Delta K
\]
da cui
\[\begin{aligned}
        \Delta K = - \Delta U \\
        \Delta E_{mec} = 0
    \end{aligned}\]
\end{document}