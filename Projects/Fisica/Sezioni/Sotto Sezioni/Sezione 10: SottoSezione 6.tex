\documentclass[../../Main/Appunti Fisica.tex]{subfiles}
\begin{document}
Una forza si definisce conservativa se soddisfa le seguenti proprietà, tra loro equivalenti.
\begin{enumerate}
    \item Il lavoro compiuto dalla forza agente su un punto materiale che si muove tra due punti, è indipendente dal percorso.
    \item Il lavoro compiuto dalla forza agente su un punto materiale che descrive una linea chiusa, è zero.
\end{enumerate}
Si ha che il lavoro compiuto da una forza conservativa è
\[
    \vb{W_{c}} = -\Delta U
\]
%
Se una forza non soddisfa nessuna delle precedenti proprietà, questa si dice non conservativa.
\\ \\
Si definisce inoltre \textit{energia meccanica} la somma di energia cinetica e potenziale, cioè
\[
    E_{mec} = U + K
\]
%
ove \(U\) comprende tutte le energie potenziali e \(K\) quelle cinetiche.
\end{document}