\documentclass[../../Main/Appunti Fisica.tex]{subfiles}
\begin{document}
Altra categoria di moti oscillatori è quella dei pendoli, di cui si tratterà solo il \textit{pendolo semplice}.
\\ \\
Si consideri la figura di seguito riportata.
\begin{figure}[!h]
    \centering
    \begin{tikzpicture}[scale = 1, every node/.style={scale=1}]
        % Ceiling & string
        \draw [pattern = north east lines] (0, 0.5) -- (0, 0) -- (3, 0) --(3, 0.5);
        \draw (1.5, 0) -- (2.5, -2); % String position at a certain time t
        \node [anchor = west] at (2, -1) {\(\va{T}\)};

        \draw [dashed] (1.5, 0) -- (1.5, -2.625); % String position whe in equilibrium
        \node [anchor = east] at (1.5, -1.315) {\(L\)};

        % Theta
        \draw (1.5, -0.5) arc(270:297.5:0.5);
        \node [anchor = north west] at (1.5, -0.5) {\(\theta\)};

        % Oscillatory path
        \begin{scope}
            \clip (0, -1.875) rectangle (3, -3);
            \draw [densely dashed] (1.5, -1.5) circle (1.125);
        \end{scope}
        \node [anchor = east] at (1, -2) {\(\vb{s}\)};


        % Mass
        \filldraw [fill = MidnightBlue] (2.5, -2) circle (0.25);
        \node [anchor = west] at (2.75, -2) {\(m\)};

    \end{tikzpicture}
    \caption{Schema modello pendolo semplice.}
    \label{fig:8}
\end{figure}

Applicando la seconda legge di Newton, poiché l'unica forza opposta all'equilibrio è la componente della forza peso, segue
\[
    \vb{F} = -m\vb{g} \sin \theta = - m \dv[2]{\vb{s}}{t}
\]
ove \(\vb{s}\) è lo spostamento misurato lungo l'arco.\\
Ma \(\vb{s} = L\theta\), e poiché \(L\) costante, segue
\[
    - m \dv[2]{\vb{s}}{t} = \dv[2]{\theta}{t} = - \frac{g}{L} \sin \theta
\]

\begin{Remark*}
    Se \(\theta \le 10\) radianti, la precedente equazione si approssima come
    \[
        \dv[2]{\theta}{t} = - \frac{g}{L} \theta
    \]
    la cui soluzione è
    \[
        \theta = \theta_{MAX} \cos(\omega t + \phi)
    \]
    ove \(\theta_{MAX}\) è il massimo spostamento angolare dall'equilibrio, con
    \[
        \omega = \sqrt{\frac{\vb{g}}{L}}
    \]
    da cui infine
    \[
        \tau = \frac{2\pi}{\omega} = 2\pi \sqrt{\frac{L}{\vb{g}}}
    \]
\end{Remark*}

\end{document}
