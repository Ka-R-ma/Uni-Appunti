\documentclass[../../Main/Appunti Fisica.tex]{subfiles}
\begin{document}
Un'altro concetto utile della termodinamica è quello di \textit{entropia}.
Questa è da considerare come misura del disordine in un sistema isolato.
\\ \\
Si consideri una trasformazione infinitesima.
Se \(\dd Q\) è la quantità di calore scambiato durante il cammino tra due stati, allora
\[
    \dd S = \frac{\dd Q}{T} \quad (\text{trasformazione reversibile})
\]
supponendo \(T\) costante.
\\ \\
Si consideri ora il caso di una trasformazione finita, poiché \(T\) non è sempre costante, segue
\[
    \Delta S = \int{i}{f}{}{S} = \int{T_{i}}{T_{f}}{\frac{1}{T}}{Q}
\]
la quale risulta essere dipendente unicamente dagli stati iniziali e finali del sistema.
\\ \\
Si consideri adesso l'entropia di una macchina di Carnot, operante a \(T_{c} \text{e} T_{h}\).
Poiché il calore è ceduto solo nelle trasformazioni isoterme, segue
\[
\Delta S  = \frac{\abs{Q_{h}}}{T_{h}} - \frac{\abs{Q_{c}}}{T_{c} = 0}   
\] 
%
Più in generale, indipendentemente dalla trasformazione
\[
\Delta S = \oint{}{}{\frac{1}{T}}{Q}    
\]
\end{document}
\clearpage