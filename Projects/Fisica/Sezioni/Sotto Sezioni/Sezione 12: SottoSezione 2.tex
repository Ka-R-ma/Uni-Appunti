\documentclass[../../Main/Appunti Fisica.tex]{subfiles}
\begin{document}
Si consideri un punto materiale su cui agisce una forza risultante \(\sum \vb{F}\) variabile nel tempo.
In accordo con quanto detto \(\sum \vb{F} = \dv{\va{a}{t}}\).
\\ \\
Segue pertanto che, se a \(t_{i}\) si ha una certa \(\vb{p_{i}}\) e a \(t_{f}\) si ha una certa \(\vb{p_{f}}\), dall'integrazione di \(\dd \va{p} = \sum \vb{F} \dd t\), segue
\[
    \Delta \va{p} = \int{t_{i}}{t_{f}}{\sum \va{F}}{t}
\]
La quantità appena trovata è definita \textit{impulso}.
\[
    \va{I} \equiv \int{t_{i}}{t_{f}}{\sum \va{F}}{t}
\]
\end{document}