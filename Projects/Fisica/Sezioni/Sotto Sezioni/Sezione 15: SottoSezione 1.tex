\documentclass[../../Main/Appunti Fisica.tex]{subfiles}
\begin{document}
Si consideri la figura di seguito riportata.
\begin{figure}[!h]
    \centering
    \begin{tikzpicture}[scale = 1, every node/.style={scale=1}]
        %TODO: schema di perturbazioni
    \end{tikzpicture}
    \caption{Schema di impulso su corda.}
    \label{fig:9}
\end{figure}

Si consideri Figura \ref{fig:9}.
Quando l'impulso viaggia lungo la corda, ogni segmento della stessa si muove in direzione perpendicolare alla direzione di propagazione.
\\
Supponendo che l'impulso si propaghi costantemente verso sinistra, indipendentemente dalla forma assunta, questa è rappresentabile come una funzione
\[
    y(x, 0) = f(x)
\]
Questa descrive lo spostamento verticale di un segmento della corda, che a \(t = 0\) è in posizione \(x\).
\\ \\
Se si considera \(\vb{v}\) la velocità dell'impulso, supponendo che la forma dello stesso non vari, segue che vi sarà uno stesso spostamento verticale per ogni \(t\) e ogni \(x\), cioè
\[
    y(x, t) = y(x - \vb{v}t, 0)
\]

Si definisce \(y\), \textit{funzione d'onda}.
\clearpage

\subsubsection{Modello: onda progressiva.}
Si consideri la figura di seguito riportata.
\begin{figure}[!h]
    \centering
    \begin{tikzpicture}[scale = 1, every node/.style={scale=1}]
        %TODO plot function + cresta, valle, lunghezza d'onda.
    \end{tikzpicture}
    \caption{TODO}
    \label{fig:10}
\end{figure}

Si definisce \textit{cresta} il massimo valore assunto, \textit{valle} il minimo.
Inoltre la distanza tra due creste è detta \textit{lunghezza d'onda}.
\\ \\
Si consideri Figura \ref{fig:10}.
Poiché sinusoidale la funzione d'onda è
\[
    y(x, 0) = A \sin x
\]
%TODO: Continuare sezione
\end{document}