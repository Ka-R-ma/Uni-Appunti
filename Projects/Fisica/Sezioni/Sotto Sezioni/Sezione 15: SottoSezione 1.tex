\documentclass[../../Main/Appunti Fisica.tex]{subfiles}
\begin{document}
Si considerino due corpi \(A, B\) posti a \textit{contatto termico}.
Questi con il tempo raggiungeranno un \textit{equilibrio termico}, cioè una situazione in cui i corpi non si scambiano energia:
né come calore né come radiazione elettromagnetica.
\\ \\
Si supponga ora che i corpi \(A. B\) non siano a contatto termico, e si supponga di avere un terzo corpo \(C\).
Si immagini di porre \(A, C\) in contatto termico, fintanto che questi raggiungano un equilibrio, annotando tale temperatura.
Si proceda analogamente per \(B\). Se una volta confrontate le temperature precedentemente annotate, si ha che queste coincidono,
allora i corpi \(A, B\) sono in equilibrio termico tra loro.
\\ \\
Quanto detto è alla base del principio zero della termodinamica, di seguito enunciato.

\begin{Principle*}[zero della termodinamica]
    Se due corpi \(A, B\) sono separatamente in equilibrio termico con un corpo \(C\), allora \(A, B\) sono in equilibrio termico tra loro.
\end{Principle*}
\end{document}
\clearpage