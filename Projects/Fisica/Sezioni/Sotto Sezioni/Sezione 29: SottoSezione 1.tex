\documentclass[../../Main/Appunti Fisica.tex]{subfiles}
\begin{document}
Si consideri il processo di carica di un condensatore.
Quando una corrente attraversa l'armatura positiva, cambia la carica su di essa, ma la corrente non attraversa lo spazio tra le due armature.
%TODO: Figura 34:1 libro

Si considerino dunque superfici \(S_{1}, S_{2}\) contornate da un percorso \(P\).
\\ \\
Da Ampere segue che
\begin{itemize}
    \item in \(S_{1}\)
          \[
              \oint{}{}{\va{B} \cdot}{\va{s}} = \mu_{0}I
          \]
          poiché percorsa da corrente.

    \item in \(S_{2}\)\[
              \oint{}{}{\va{B} \cdot}{\va{s}} = 0
          \]
          poiché non attraversa da alcuna corrente.
\end{itemize}
Ma ciò è una contraddizione, poiché si avrebbe in tal modo una discontinuità di corrente.
\\ \\
Maxwell risolse tale problema postulando che, all'equazione di Ampere andasse aggiunta la \textit{corrente di spostamento} \(I_{s}\), definita come
\[
    I_{s} \equiv \varepsilon_{0} \dv{\Phi_{E}}{t}
\]
Da cui la legge di Ampere generalizzata è
\[
    \oint{}{}{\va{B} \cdot}{\va{s}} = \mu_{0}I + \varepsilon_{0}\mu_{0}\dv{\Phi_{E}}{t}
\]
\end{document}