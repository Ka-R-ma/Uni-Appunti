\documentclass[../../Main/Appunti Fisica.tex]{subfiles}
\begin{document}
Si consideri un solenoide ideale, applicando Ampere, poiché ideale, il solenoide presenterà un campo \(\va{B}\) solo al suo interno, segue
\[
    \oint{}{}{\va{B}}{\va{s}} = \vb{B} \int{}{}{}{s} = \vb{B}l
\]
ove \(l\) è la lunghezza del tratto di solenoide considerato.
\\ \\
Supponendo che \(n = N / l\) sia il numero di spire per unità di lunghezza, segue
\[\begin{gathered}
        \oint{}{}{\va{B}}{\va{s}} = \vb{B}l = \mu_{0}NI   \\
        \vb{B} = \mu_{0} \frac{N}{l} I = \mu_{0} n I \\
    \end{gathered}\]
ove \(N\) è il numero di spire nel tratto \(l\).
\end{document}