\documentclass[../../Main/Appunti Fisica.tex]{subfiles}
\begin{document}
Per calcolare il potenziale di una carica puntiforme ad una distanza \(r\), si consideri l'Equazione \eqref{eq:16}.
\[\begin{aligned}
        \Delta V & = - \int{\textcircled{a}}{\textcircled{b}}{\va{E} \cdot}{\va{s}}                       \\
                 & = - \int{\textcircled{a}}{\textcircled{b}}{K_{C} \frac{q}{r^{2}} \vu{r} \cdot}{\va{s}}
    \end{aligned}\]
ma \(\vu{r} \cdot \dd \va{s} = \dd \va{s} \cos \theta = \dd r\), segue
\[
    \Delta V = - K_{C} q \int{\textcircled{a}}{\textcircled{b}}{\frac{1}{r^{2}}}{r} = K_{C} q \left[ \frac{1}{r_{\textcircled{b}}} - \frac{1}{r_{\textcircled{a}}} \right]
\]
\end{document}