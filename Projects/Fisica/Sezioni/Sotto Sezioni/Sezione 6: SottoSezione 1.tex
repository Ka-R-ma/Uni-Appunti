\documentclass[../../Main/Appunti Fisica.tex]{subfiles}
\begin{document}
Si consideri la figura di seguito riportata.
\begin{figure}[!h]
    \centering
    \begin{tikzpicture}[scale = 1, every node/.style={scale=1}]

        % Axis
        \draw[-stealth, thick] (0, 0) -- (0, 3);
        \node (y) [anchor = west] at (0, 3) {\(\mathit{y}\)};

        \draw[-stealth, thick] (0, 0) -- (4, 0);
        \node (x) [anchor = south] at (4, 0) {\(\mathit{x}\)};

        % Bullet trajectory
        \draw (1.5, 1.5) parabola (0.5, 0.825);
        \draw (1.5, 1.5) parabola (3, 0);

        % Theta
        \draw (0.5, 0.5) arc (45:-15:0.5) (0.5, 0);
        \draw (0.5, 0.5) arc (45:65:0.5);
        \node [anchor = west] at (0.5, 0.5) {\(\theta\)};

        % Points of interest
        \node [circ, color = MidnightBlue] at (0, 0) {};
        \node (a) [anchor = north east] at (0, 0) {\textcircled{a}};

        \node [circ, color = MidnightBlue] at (1.5, 1.5) {};
        \node (b) [anchor = south east] at (1.5, 1.5) {\textcircled{b}};

        \node [circ, color = MidnightBlue] at (3, 0) {};
        \node (c) [anchor = north east] at (3, 0) {\textcircled{c}};

        % Velocity in A
        \draw[-to, MidnightBlue, thick] (0, 0) -- (0.5, 0.825);
        \node [anchor = south] at (0.5, 0.825) {\(\va{v_{0}}\)};

        % Maximum height and range
        %% Height
        \draw[to-to, thick, MidnightBlue] (1.5, 1.5) -- (1.5, 0);
        \node (h) [anchor = west] at (1.5, 0.75) {\(H\)};

        %% Range
        \draw[|-|, thick, MidnightBlue] (0, -1) -- (3, -1);
        \node (r) [anchor = south] at (1.5, -1) {\(R\)};

    \end{tikzpicture}
    \caption{Schema rappresentativo di moto del proiettile con gittata e altezza.}
    \label{fig:2}
\end{figure}

Dalla \textit{Figura \ref{fig:2}}, supponendo che in \textcircled{a}, \(x_{0} = 0\) e il proiettile ha una certa velocità \(\vb{v_{y}}\), si nota che:
\begin{itemize}
    \item \textcircled{b} è il punto di massima elevazione raggiunto dal proiettile;
    \item \textcircled{c} è il punto in cui il proiettile raggiunge lo stesso livello orizzontale di partenza.
\end{itemize}
%
Ponendo \(H \text{e} R\) rispettivamente: distanza massima verticale, distanza massima orizzontale, segue
\[\begin{gathered}
        H = \frac{\vb{v_{0}}^{2} \sin^{2}{\theta_{0}}}{2\vb{g}} \\
        R = \vb{v_{x}} \cos{\theta_{0}} \frac{\vb{v_{x}}^{2} \sin{2\theta_{0}}}{\vb{g}} \\
    \end{gathered}\]
%
Pertanto si ha che \(R\) è massima se \(\theta_{0} = 45\).
\end{document}