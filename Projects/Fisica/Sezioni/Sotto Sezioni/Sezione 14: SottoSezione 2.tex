\documentclass[../../Main/Appunti Fisica.tex]{subfiles}
\begin{document}
Si consideri ora il sistema in Figura \ref{fig:6}, dal punto di vista energetico.
Poiché un sistema isolato, ci si aspetta che l'energia meccanica sia costante.
Dalle equazioni \eqref{eq:5} e \eqref{eq:6}, segue
\[\begin{gathered}
        K = \frac{1}{2} m v^{2} = \frac{1}{2} m \omega^{2} A^{2} \sin^{2}(\omega t + \phi) \\
        U = \frac{1}{2} k x^{2} = \frac{1}{2} k \omega^{2} A^{2} \cos^{2}(\omega t + \phi) \\
    \end{gathered}\]
%
Poiché si \(K \text{sia} U\) sono sempre non nulle, segue
\[\begin{aligned}
        E = K + U & = \frac{1}{2} k A^{2}\left[ \sin^{2}(\omega t + \phi) + \cos^{2}(\omega t + \phi) \right] \\
                  & = \frac{1}{2} k A^{2}
    \end{aligned}\]

Infine, la velocità si ricava come
\[
    \va{v} = \pm \sqrt{\frac{k}{m} (A^{2} - x^{2})} = \omega \sqrt{(A^{2} - x^{2})}
\]
\end{document}