\documentclass[../../Main/Appunti Fisica.tex]{subfiles}
\begin{document}
Le equazioni di Maxwell sono
\begin{equation}\label{eq:23}
    \oint{}{}{\va{E} \cdot}{\va{A}} = \frac{q}{\varepsilon_{0}} \quad \text{Legge di Gauss.}
\end{equation}

\begin{equation}\label{eq:24}
    \oint{}{}{\va{B} \cdot}{\va{A}} = 0 \quad \text{Legge di Gauss per il magnetismo.}
\end{equation}

\begin{equation}\label{eq:25}
    \oint{}{}{\va{E} \cdot}{\va{s}} = -\dv{\Phi_{E}}{t} \quad \text{Legge di Faraday.}
\end{equation}

\begin{equation}\label{eq:26}
    \oint{}{}{\va{B} \cdot}{\va{s}} = \mu_{0}I + \varepsilon_{0}\mu_{0}\dv{\Phi_{E}}{t} \quad \text{Legge di Ampere-Maxwell.}
\end{equation}

Dunque sintetizzando i paragrafi precedenti:
\begin{itemize}
    \item L'Equazione \eqref{eq:23} stabilisce che:
          il flusso elettrico totale uscente da una qualsiasi superficie chiusa è dato dalla carica totale diviso \(\varepsilon_{0}\).
    \item L'Equazione \eqref{eq:24} stabilisce che:
          il flusso magnetico totale da una superficie chiusa, è nullo.
    \item  L'Equazione \eqref{eq:25} stabilisce che:
          la \fem indotta è uguale a alla derivata rispetto al tempo, cambiata di segno, del flusso magnetico su una superficie qualsiasi che ha un cammino come contorno.
    \item  L'Equazione \eqref{eq:26} stabilisce che: l'integrale di linea del campo magnetico, su un cammino chiuso,
          è la somma di \(\mu_{0}\) volte la corrente totale e \(\varepsilon_{0}\mu_{0}\) volte la derivata rispetto al tempo del flusso elettrico.
\end{itemize}
\end{document}