\documentclass[../../Main/Appunti Fisica.tex]{subfiles}
\begin{document}
Spesso capita di non poter semplificare un circuito complesso ad un semplice chiuso unico.
La procedura di analisi di tali circuiti è di molto semplificata se si applicano i principi (o leggi) di Kirchhoff.\\
Queste sono
\begin{itemize}
    \item \underline{legge dei nodi}: in un nodo la somma delle correnti deve essere nulla.
          \[
              \sum\limits_{nodo} I = 0
          \]

    \item \underline{legge delle maglie}: in una maglia la somma delle tensioni deve essere nulla.
          \[
              \sum\limits_{maglia} \Delta V = 0
          \]
\end{itemize}
\end{document}