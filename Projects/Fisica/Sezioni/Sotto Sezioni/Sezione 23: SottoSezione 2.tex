\documentclass[../../Main/Appunti Fisica.tex]{subfiles}
\begin{document}
Si consideri un reticolo di atomi liberi.
Quando si applica un campo elettrico, tali elettroni seguiranno una deriva nella stessa direzione del campo ma con verso opposto,
con un velocità \(\vb{v_{d}}\) molto piccola.
\\ \\
Considerando le seguenti ipotesi, si deriverà il modello.
\begin{enumerate}
    \item Il moto dopo l'urto è indipendente dal moto antecedente lo stesso.
    \item L'energia tra un urto e l'altro di un elettrone nel campo viene interamente ceduta agli atomi del conduttore quando l'elettrone urta con essi.
\end{enumerate}

Quando l'elettrone di massa \(m_{e}\) e carica \(- e\) è soggetto ad un campo elettrico \(\va{E}\), risente di una forza \(\va{F} = - e \va{E}\), da Newton segue
\[
    \va{a} = \frac{\sum \vb{F}}{m} = \frac{q \va{E}}{m_{e}} \qquad \text{con} q = -e
\]
%
Poiché \(\va{E}\) è costante, dopo l'urto se \(\va{v_{i}}\) è la velocità immediatamente successiva allo stesso, la velocità finale sara
\[
    \va{v_{f}} = \va{v_{i}} + \va{a} t = \va{v_{i}} + \frac{q \va{E}}{m_{e}} t
\]
ma poiché tale velocità si ripete in un periodo \(\tau\), si ha
\[
    \mv{v_{f}} = \va{v_{d}} = \frac{q \va{E}}{m_{e}} \tau
\]
dunque
\[\begin{gathered}
        J = nq\vb{v_{d}} = \frac{nq^{2}\vb{E}}{m_{e}}\tau \\
        \sigma = \frac{nq^{2}}{m_{e}}\tau \\
        \rho = \frac{1}{\sigma} =  \frac{m_{e}}{nq^{2}\tau} \\
    \end{gathered}\]
\end{document}
\clearpage