\documentclass[../../Main/Appunti Fisica.tex]{subfiles}
\begin{document}
Si consideri un'area dello spazio in cui è presente una carica distribuita \(q\), si consideri di tale carica una porzione \(\Delta q\).
Calcolando il campo elettrico di tale carica, segue
\[
    \Delta \va{E} = \frac{K_{C} \Delta q}{r^{2}} \vu{r}
\]
ove \(r\) è la distanza tra \(\Delta q\) è un punto qualsiasi.\\
Per cui il campo elettrico totale in un punto \(P\) è
\[
    \va{E} \approx K_{C} \sum\limits_{i} \frac{\Delta q_{i}}{r_{i}^{2}} \vu{r}
\]
da cui se \(\Delta q_{i} \to 0\), segue
\[
    \va{E} =  \lim\limits_{\Delta q_{i} \to 0} \sum\limits_{i} K_{C} \frac{\Delta q_{i}}{r_{i}^{2}} \vu{r} = K_{C} \frac{\dd q}{r^{2}} \vu{r}
\]
%
Per comodità nei calcoli si usa la seguente notazione.
\begin{Notation*}
    \[\begin{gathered}
            \rho \equiv \frac{Q}{V}    \\
            \sigma \equiv \frac{Q}{A}  \\
            \lambda \equiv \frac{Q}{l} \\
        \end{gathered}\]
\end{Notation*}
le quali sono rispettivamente \textit{densità di carica di volume, di superficie, di linea}, \(Q\) è una carica distribuita.
\\ \\
Nel caso di cariche infinitesimal, valgono
\[\begin{gathered}
    \dd q = \rho \dd V \\
    \dd q = \sigma \dd A \\
    \dd q = \lambda \dd l \\
\end{gathered}\]
\end{document}
\clearpage