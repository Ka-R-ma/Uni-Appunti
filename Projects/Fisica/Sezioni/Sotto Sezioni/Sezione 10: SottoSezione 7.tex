\documentclass[../,,/Main/Appunti Fisica.tex]{subfiles}
\begin{document}
Si è detto precedentemente che il lavoro compiuto da una forza conservativa, dipende unicamente dalle coordinate iniziali e finali.
\`E pertanto possibile definire una funzione energia potenziale \(U\), tale che il lavoro compiuto sugli elementi del sistema sia uguale alla variazione di energia potenziale.
\[\begin{aligned}
        \vb{W_{c}} & = \int{x_{i}}{x_{f}}{Fx}{x}
                   & = -\Delta U
    \end{aligned}\]
Se il punto di applicazione della forza subisce uno spostamento infinitesimale \(\dd x\), si può può esprimere la variazione infinitesimale dell'energia potenziale de sistema nella forma
\[
    \dd U = - \vb{F_{x}} \dd x
\]
Segue dunque
\[
    \vb{F_{x}} = \dv{v}{x}
\]
\end{document}