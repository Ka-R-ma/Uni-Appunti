\documentclass[../../Main/Appunti Fisica.tex]{subfiles}
\begin{document}
Molti dei termini utilizzati sin'ora, hanno pressoché il medesimo significato di quello assegnatogli nella vita quotidiana.
Si procede ora all'introduzione di una nuova grandezza fisica, quale il \textit{lavoro}.
%
Si consideri una forza costante applicata ad un corpo, si definisce lavoro il prodotto tra il modulo della forza, l'eventuale spostamento causato ed il coseno dell'angolo compreso tra la forza e lo spostamento,
cioè
\[
    \vb{W} \equiv \vb{F} \Delta \vb{s} \cos \theta
\]
%
Nel SI, il lavoro è misurato in Joule, (\SI{}{\joule}).
\begin{center}
    \SI[per-mode = fraction, unit-color = MidnightBlue]{1}{\joule \equiv \kilogram \cdot \metre \tothe{2} \per \second \tothe{2}}
\end{center}
\end{document}