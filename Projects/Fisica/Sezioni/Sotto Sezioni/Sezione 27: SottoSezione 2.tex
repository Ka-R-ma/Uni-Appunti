\documentclass[../../Main/Appunti Fisica.tex]{subfiles}
\begin{document}
Dalla legge di Faraday
\[
    \varepsilon_{ind} = \dv{\Phi_{B}}{t}
\]
si nota che la \fem indotta e il campo magnetico hanno segni opposti. \\
Questo concetto, per quanto banale, ha un significato fisico molto importante, espresso nella legge di Lenz.
\begin{Law*}[di Lenz]
    La polarità della \fem indotta è tale da produrre una corrente che genera un flusso magnetico che
    si oppone alla variazione del flusso che attraversa l'area racchiusa dalla spira di corrente.
\end{Law*}
\end{document}