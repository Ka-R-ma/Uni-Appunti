\documentclass[../../Main/Appunti Fisica.tex]{subfiles}
\begin{document}
Dati due o più condensatori, questi possono tra loro essere collegati
\begin{enumerate}
    \item in serie;
    \item in parallelo.
\end{enumerate}

\subsubsection{Condensatori in serie.}
Dati due condensatori, questi si dicono in serie se, la carica \(Q\) sul primo condensatore è la stessa carica presente sul secondo condensatore.
\\ \\
Segue dunque che la differenza di potenziale fornita, si distribuisca su ciascun condensatore, pertanto valore
\[
    \Delta V = \Delta V_{1} + \Delta V_{2}
\]
%
Supponendo di voler utilizzare un'unico condensatore, tale che questi sia equivalente ai due precedentemente presenti, dall'Equazione \eqref{eq:17} segue
\[
\Delta V = \frac{Q}{C_{eq}}
\]
da cui pertanto
\[\begin{gathered}
    \frac{Q}{C_{eq}} = \frac{Q_{1}}{C_{1}} + \frac{Q_{2}}{C_{2}} \qquad \text{ma} Q = Q_{1} = Q_{2} \\
    \frac{1}{C_{eq}} = \frac{1}{C_{1}} + \frac{1}{C_{2}} \\
\end{gathered}\]

\subsubsection{Condensatori in parallelo.}
Dati due condensatori, questi si dicono in parallelo se, la differenza di potenziale ai capi del primo condensatore è la stessa presente sul secondo.
cioè
\[
    \Delta V = \Delta V_{1} = \Delta V_{2}
\]
%
Supponendo di voler utilizzare un'unico condensatore, tale che questi sia equivalente ai due precedentemente presenti, dall'Equazione \eqref{eq:17} segue
\[
Q = C_{eq} \Delta V    
\]
da cui pertanto
\[\begin{gathered}
    C_{eq} \Delta V = C_{1} \Delta V_{1} + C_{2} \Delta V_{2} \qquad \text{ma} \Delta V = \Delta V_{1} = \Delta V_{2} \\
    C_{eq} = C_{1} + C_{2}
\end{gathered}\]
\end{document}