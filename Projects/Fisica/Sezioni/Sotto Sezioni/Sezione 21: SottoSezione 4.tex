\documentclass[../../Main/Appunti Fisica.tex]{subfiles}
\begin{document}
    Dall'Equazione \eqref{eq:16}, si può esprimere la differenza di potenziale \(\dd V\) tra due punti come
    \[
    \dd V = \va{E} \dd \va{s}    
    \]
    Se la distribuzione di carica che genera \(\va{E}\) ha una simmetria sferica, segue \(\va{E} \cdot \dd \va{s} = \vb{E} \dd r\), da ciò
    \[
    \vb{E_{r}} = \dv{V}{r}   
    \]
    Se \(\va{E}\) si espande in ogni direzione, vale allora quanto segue
    \[
    \vb{E_{x}} = \pdv{V}{x}, \quad \vb{E_{y}} = \pdv{V}{y}, \quad \vb{E_{z}} = \pdv{V}{z},
    \]
\end{document}
\clearpage