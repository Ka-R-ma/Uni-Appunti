\documentclass[../../Main/Appunti Fisica.tex]{subfiles}
\begin{document}
A seguito di osservazioni sperimentali, Biot e Savart formularono un'equazione per il calcolo del campo magnetico in un dato punto dello spazio.
Questa risulta essere la seguente
\[
    \dd \va{B} = \frac{\mu_{0}}{4 \pi} \frac{I \dd \va{s} \cp \vu{r}}{r^{2}}
\]
ove \(\mu_{0}\) è la \textit{costante di permeabilità del vuoto}, pari a \(4 \pi \cp 10^{-7}\)\SI[unit-color = MidnightBlue, per-mode = fraction]{}{\tesla \cdot \metre \per \ampere}.
Segue da tale equazione che il campo magnetico totale sia
\[
    \vb{B} = \frac{\mu_{0} I}{4 \pi} \oldInt{\frac{\dd \va{s} \cp \vu{r}}{r^{2}}}
\]
\end{document}