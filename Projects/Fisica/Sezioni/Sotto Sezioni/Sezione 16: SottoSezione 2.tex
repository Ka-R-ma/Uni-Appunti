\documentclass[../../Main/Apputni Fisica.tex]{subfiles}
\begin{document}
La \textit{capacità termica} di una particolare sostanza, è definita come la quantità di energia utile ad aumentare di un grado la temperatura della stessa.
Segue che, se \(Q\) causa un variazione \(\Delta T\) nella temperatura, allora
\[
    Q = C \Delta T
\]
%
Si definisce \textit{calore specifico}, la capacità termica per unita di massa. Cioè
\[
    c = \frac{Q}{m \Delta T}
\]
%
da cui
\[
    Q = m c \Delta T
\]
\end{document}