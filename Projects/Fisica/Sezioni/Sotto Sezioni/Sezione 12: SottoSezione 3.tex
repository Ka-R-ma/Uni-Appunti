\documentclass[../../Main/Appunti Fisica.tex]{subfiles}
\begin{document}
Si definisce urto l'evento in cui due, o più, particelle interagiscono per un breve istante tramite forze.
Questi si suddividono in urti
\begin{itemize}
    \item \underline{elastici}: se l'urto non causa una variazione dell'energia cinetica;
    \item \underline{anelastici}: se l'urto causa una variazione dell'energia cinetica;
          \begin{itemize}
              \item \underline{semplice}: se ne consegue che i corpi rimangono separati;
              \item \underline{completamente anaelastico}: se ne consegue che i corpi rimangono uniti.
          \end{itemize}
\end{itemize}
\clearpage

\subsubsection{Urti completamente anelastici.}
Si consideri Figura \ref{fig:7}.
I due corpi urtano centralmente e rimanendo poi uniti.
Poiché il sistema è isolato
\[
    m_{1}\va{v_{1}} + m_{2}\va{v_{2}} = (m_{1} + m_{2})\vb{v_{f}}
\]
da cui quindi
\[
    \vb{v_{f}} = \frac{m_{1}\va{v_{1}} + m_{2}\va{v_{2}}}{m_{1} + m_{2}}
\]

\subsubsection{Urti elastici.}
Si consideri nuovamente Figura \ref{fig:7}.
I due corpi urtano e dopo si allontanano con due velocità differenti.
Assumendo l'urto anelastico, poiché per un infinitesimo i due corpi risultano uniti, sia \(\va{p}\) sia \(K\) sono costanti.
Segue pertanto
\begin{equation}\label{eq:2}
    m_{1}\va{v_{1i}} + m_{2}\va{v_{2i}} = m_{1}\va{v_{1f}} + m_{2}\va{v_{2f}}
\end{equation}
da cui quindi
\begin{equation}\label{eq:3}
    m_{1}(\va{1f} - \va{v_{1i}}) = m_{2}(\va{2f} - \va{v_{2i}})
\end{equation}
Ponendo ora a sistema l'Equazioni \eqref{eq:2} e \eqref{eq:3}, segue
\[\begin{gathered}
        \va{v_{1f}} = \left(\frac{m_{1} - m_{2}}{m_{1} + m_{2}} \right) \va{v_{1i}} + \left(\frac{2 m_{2}}{m_{1} + m_{2}} \right) \va{v_{2i}}\\
        \va{v_{2f}} = \left(\frac{2 m_{1}}{m_{1} + m_{2}}\right) \va{v_{1i}}  + \left(\frac{m_{1} - m_{2}}{m_{1} + m_{2}} \right) \va{v_{2i}}
    \end{gathered}\]
\end{document}