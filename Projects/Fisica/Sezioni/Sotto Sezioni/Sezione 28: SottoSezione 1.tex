\documentclass[../../Main/Appunti Fisica.tex]{subfiles}
\begin{document}
Un elemento con grande induttanza è detto \textit{induttore}.
\\ \\
Si consideri la figura di seguito riportata.
\begin{figure}[!h]
    \centering
    \begin{tikzpicture}[scale = 1, every node/.style={scale=1}]

        \node (source) [battery1shape, rotate = -90] at (0, 0) {};
        \node [anchor = east] at (-0.375, 0) {\(\varepsilon\)};

        \node (sA) [circ] at ($(source.left) + (0.5, 1.25)$) {};
        \node [anchor = south] at ($(source.left) + (0.5, 1.25)$) {\(a\)};

        \node (sB) [circ] at ($(source.left) + (0.75, 0.75)$) {};
        \node [anchor = east] at ($(source.left) + (0.75, 0.75)$) {\(b\)};

        \node (sC) [circ] at ($(sA) + (0.5, 0)$) {};
        \node [anchor = south] at ($(sA) + (0.5, 0)$) {\(s\)};

        \draw (source.left) to [ccgsw] ($(source.left) + (0, 1.25)$) to (sA)
        (sC) to [R] ($(sC) + (1.75, 0)$) to [L] ($(source.right) + (2.75, -1.25)$) to ($(source.right) + (0, -1.25)$) to (source.right);

        \draw (sB) to ($(source.right) + (0.75, -1.25)$);

        \draw (sC) to ($(sC) - (0.5, 0.25)$);

    \end{tikzpicture}
    \caption{Circuito RL.}
    \label{fig:21}
\end{figure}

Si supponga \(s\) chiuso in \(a \text{a} t = 0\), mentre l'altro interruttore è aperto quando \(t < 0\).
Applicando il secondo principio di Kickoff
\begin{equation}\label{eq:20}
    \varepsilon  - IR - \dv{I}{t}L = 0
\end{equation}

Ponendo \(x = (\varepsilon / R) - I \text{e} \dd x = -\dd I\), integrando segue
\[\begin{gathered}
        \int{x_{i}}{x_{f}}{\frac{1}{x}}{x} = \frac{R}{L} \int{0}{t}{}{t} \\
        \ln \left( \frac{x}{x_{0}} \right) = \frac{R}{L} t
    \end{gathered}\]
da cui prendendo l'antilogaritmo
\[
    x = e^{-Rt/L} x_{0}
\]

Dunque a \(t = 0, I = 0\), ma dalla definizione di \(X, x_{0} = \varepsilon / R\), da cui
\[
    \frac{\varepsilon}{R} - I = \frac{\varepsilon}{R} e^{-Rt / L}
\]
cioè
\[
    I = \frac{\varepsilon}{R} \left( 1 - e^{-Rt / L} \right) = \frac{\varepsilon}{R} \left( 1 - e^{-t / \tau} \right)
\]
ponendo \(\tau = L / R\).
\end{document}
