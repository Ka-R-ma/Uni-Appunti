\documentclass[../../Main/Appunti Fisica.tex]{subfiles}
\begin{document}
Ogni qual volta si è in presenza di una transizione di fase, cioè il passaggio da uno stato ad un altro, lo scambio di calore non comporta una variazione di temperatura.
\\ \\
In questi casi, se \(Q\) è necessaria al cambiamento di stato di una massa \(m\) di una certa sostanza, si viene ad avere il \textit{calore latente} \(L\).
\[
    L \equiv \frac{Q}{m}
\]
segue dunque
\[
    Q = \pm mL
\]
Quest'ultima equazione stabilisce che se il sistema assorbe calore, \(Q\) è positivo, se invece lo cede allora \(Q\) sarà negativo.
\end{document}
\clearpage