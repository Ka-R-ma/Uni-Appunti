\documentclass[../../Main/Appunti Fisica.tex]{subfiles}
\begin{document}
Si consideri un corpo che attraversa un mezzo viscoso, come ad esempio l'aria.
Questi risentirà di una resistenza dovuta all'iterazione stessa tra corpo e mezzo; tale resistenza è definita \textit{forza d'attrito}.
\\ \\
Si supponga di applicare una forza ad un corpo, se la forza è sufficientemente piccola il corpo rimarrà fermo.
Segue dalla terza legge di Newton, che il corpo esercita una forza in resistenza a quella applicata; si definisce tale forza \textit{forza d'attrito statico}.
\\ \\
Si supponga ora di aumentare progressivamente la forza applicata, segue che dopo un certo periodo di tempo la forza d'attrito statico raggiunge il suo limite,
oltre il quale il corpo comincia a muoversi. Sebbene in movimento, il corpo è ancora soggetto ad una forza, la quale è definita \textit{forza d'attrito dinamico}.
\\ \\
Osservazioni sperimentali hanno dimostrato che sia la forza di attrito statico sia quella di attrito dinamico, dipendo dalla forza normale.
Più precisamente
\begin{itemize}
    \item la forza di attrito statico risulta essere
          \[\vb{F_{s}} \le \mu_{s}\vb{N}\]
          ove \(\mu_{s}\) è coefficiente di attrito statico.

    \item la forza di attrito dinamico risulta essere
          \[\vb{F_{d}} \le \mu_{k}\vb{N}\]
          ove \(\mu_{k}\) è coefficiente di attrito dinamico.
\end{itemize}

\begin{Remark*}
    Sia \(\mu_{k} \text{che} \mu_{s}\) dipendono dal materiale che applica resistenza.
\end{Remark*}
\end{document}