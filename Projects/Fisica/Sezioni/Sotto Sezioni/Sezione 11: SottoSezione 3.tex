\documentclass[../../Main/Appunti Fisica.tex]{subfiles}
\begin{document}
Si consideri un corpo che scivola lungo una superficie. Si supponga questi faccia parte di un sistema che subisce variazione di energia potenziale di qualche tipo.
Per il principio di conservazione, segue che l'energia potenziale dissipata è stata convertita in energia cinetica, da cui
\[
    \Delta E_{mec} = \Delta K + \Delta U = - \vb{F_{k}}\dd
\]
%
Se il sistema su cui agiscono le forze non conservative, è un sistema non isolato, allora
\[
    \Delta E_{mec} = - \vb{F_{k}} \dd + \sum \vb{W_{e}}
\]
\end{document}