\documentclass[../../Main/Appunti Fisica.tex]{subfiles}
\begin{document}
Si consideri la figura di seguito riportata.
\begin{figure}[!h]
    \centering
    \begin{minipage}{0.45 \textwidth}
        \begin{tikzpicture}[scale = 0.75, every node/.style={scale=1}]
            % Magnetic field
            \foreach \i in {2,3, ..., 6}{
                    \foreach \j in {2, 3, ..., 6}{
                            \node [] at (0.5 * \i, 0.5 * \j) {\(\cp\)};
                        }
                }

            % Circuit
            \draw [very thick] (1, 1) to [R] (1, 3);

            \filldraw (1, 1) rectangle (3.25, 1.125);
            \filldraw (1, 3) rectangle (3.25, 3.125);

            \draw [pattern = north east lines] (2.75, 1) rectangle (3, 3.125);

            % Forces and auxiliary notes
            \draw [|-|, thick] (0, 1) -- (0, 3);
            \node [anchor = east] at (0, 2) {\(l\)};

            \draw [|-|, thick] (1, 0) -- (3.25, 0);
            \node [anchor = north] at (2, 0) {\(x\)};

            \draw [-stealth, thick, color = MidnightBlue] (2.875, 2.0625) -- (2.375, 2.0625);
            \node [anchor = south] at (2.375, 2.0625) {\(\va{F_{B}}\)};

            \draw [-stealth , thick, color = MidnightBlue] (2.875, 1.5625) -- (3.375, 1.5625);
            \node [anchor = north] at (3.875, 1.5625) {\(\va{F_{app}}\)};

            \draw [-stealth, thick, color = MidnightBlue] (2.875, 2.5625) -- (3.375, 2.5625);
            \node [anchor = south] at (3.875, 2.5625) {\(\va{v}\)};
        \end{tikzpicture}
    \end{minipage}
    \begin{minipage}{0.45 \textwidth}
        \begin{tikzpicture}[scale = 0.75, every node/.style={scale=1}]

            \node (source) [battery1shape, rotate = -90] at (3, 0) {};

            \draw (source.left) to ($ (source.left) + (0, 1)$) to ($ (source.left)  + (-2, 1)$);
            \draw (source.right) to ($ (source.right) + (0, -1)$) to ($ (source.right)  + (-2, -1)$);
            \draw ($ (source.left)  + (-2, 1)$) to [R] ($ (source.right)  + (-2, -1)$);

            \draw [-stealth, thick] (2, 1.5) -- (1, 1.5);
            \node [anchor = west] at (2, 1.5) {\(I\)};

        \end{tikzpicture}
    \end{minipage}
    \caption{Circuito con barretta mobile.}
    \label{fig:19}
\end{figure}

Se alla barretta è applicata una forza \(\va{F_{app}}\), questa si muoverà con una velocità \(\va{v}\),
facendo si che le cariche libere della barretta siano soggette ad una forza magnetica lungo la stessa.
\\ \\
Quindi in un dato momento il flusso magnetico sarà
\[
    \Phi_{B} = \vb{B} l x
\]
da cui, applicando Faraday, segue
\[
    \varepsilon_{ind} = - \dv{\Phi_{B}}{t} = - \dv{}{t} (\vb{B}lx) = - \vb{B} l v
\]
da cui la corrente indotta è
\[
    I_{ind} = \frac{\abs{\varepsilon_{ind}}}{R} = \frac{- \vb{B} l v}{R}
\]
\end{document}