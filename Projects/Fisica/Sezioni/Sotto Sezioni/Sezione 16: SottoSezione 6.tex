\documentclass[../../Main/Appunti Fisica.tex]{subfiles}
\begin{document}
Le trasformazioni termodinamiche si suddividono in
\begin{itemize}
    \item \underline{adiabatiche}: l'energia non è scambiata come calore, cioè \(Q = 0\).
          Per il primo principio segue
          \[
              \Delta E_{int} = \vb{W}
          \]

    \item \underline{isobare}: i valori di lavoro e calore sono entrambi diversi da zero.
          Segue che il lavoro sul gas è
          \[
              \vb{W} = -P \Delta V
          \]
          ove \(P\) è la pressione del gas, costante per l'intera trasformazione.

    \item \underline{isocore}: il volume del gas non varia, segue che il lavoro è nullo.
          Per il primo principio segue
          \[
              \Delta E_{int} = Q
          \]

    \item \underline{isoterma}: la temperatura rimane costante.
\end{itemize}

\subsubsection{Espansione isoterma di un gas perfetto.}
Si consideri Figura \ref{fig:9}.
Dall'applicazione dell'Equazione \eqref{eq:9}, segue
\[\begin{aligned}
    \vb{W} &= \int{V_{i}}{V_{f}}{-\vb{P}}{V} = \int{V_{i}}{V_{f}}{- \frac{nRT}{V}}{V}\\
    &= - nRT \int{V_{i}}{V_{f}}{\frac{1}{V}}{V} \\
    &= - nRT \ln(V)_{V_{i}}^{V_{f}} \\
    &= - nRT \ln\left(\frac{V_{i}}{V_{f}}\right)
\end{aligned}\]
\end{document}