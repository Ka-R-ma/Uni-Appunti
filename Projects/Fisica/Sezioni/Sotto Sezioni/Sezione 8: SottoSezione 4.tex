\documentclass[../../Main/Appunti Fisica.tex]{subfiles}
\begin{document}
Si supponga di applicare una forza su un corpo di massa \(m\), in un certo senso, il corpo applicherà una resistenza.
Da un'esperienza simile, Newton dedusse quanto poi sarebbe diventata la terza legge.

\begin{Law*}[terza di Newton]
    Se due corpi interagiscono l'uno con l'altro, la forza \(\vb{F_{1, 2}}\) esercitata sul corpo due dal corpo uno,
    risulta essere uguale in modulo e opposta in verso alla forza \(\vb{F_{2, 1}}\) esercitata sul corpo uno dal corpo due.
    \\
    Cioè
    \[
        \vb{F_{1, 2}} = - \vb{F_{2, 1}}
    \]
\end{Law*}
\end{document}