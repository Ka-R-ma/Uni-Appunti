\documentclass[../../Main/Appunti Fisica.tex]{subfiles}
\begin{document}
Ogni corpo sulla terra è soggetto ad una forza di attrazione verso il basso, la quale è definita come forza peso \(\vb{F_{p}}\).
\\ \\
Si può pertanto definire ``peso'', l'intensità con la quale un corpo subisce tale forza.
\\ \\
Si consideri un corpo di massa \(m\) in caduta libera, come detto questi è soggetto, per semplicità, unicamente alla forza peso,
quindi
\[
    \sum \vb{F} = \vb{F_{p}} = m\vb{a}
\]
da osservazioni sperimentali si osserva che l'accelerazione a cui i corpi in caduta libera sono soggetti è costante ed è pari a \SI[unit-color = MidnightBlue]{9,80}{\metre \cdot \second},
che per la sua importanza è indicata con \(\vb{g}\).

\begin{Remark*}
    Nei presenti appunti l'accelerazione sarà considerata positiva se diretta verso il basso o verso destra, negativa in caso contrario.
\end{Remark*}
\end{document}