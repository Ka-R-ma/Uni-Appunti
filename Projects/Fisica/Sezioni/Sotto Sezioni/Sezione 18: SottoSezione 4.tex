\documentclass[../../Main/Appunti Fisica.tex]{subfiles}
\begin{document}
Ricordando che l'entropia di sistemi reali dipende solo dagli stati iniziali e finali del sistema,
si può calcolare l'entropia di una trasformazione irreversibile, in stato di equilibrio, fissando una trasformazione reversibile tra gli stessi stai e calcolando \(\Delta S\).
\\ \\
Poiché la variazione di entropia, se il processo è irreversibile, è sempre positiva, si può enunciare il secondo principio nel seguente modo.
\begin{Principle*}[secondo della termodinamica legato all'entropia]
    L'entropia di un sistema isolato che si evolve non può mai diminuire.
\end{Principle*}

\subsubsection{Variazione di entropia nella conduzione di calore.}
Si considerino una sorgente calda ed una fredda a contatto termico, con cui si ha un trasferimento \(Q\) di calore.
Poiché la sorgente fredda assorbe \(Q, \Delta S_{c} = Q / T_{c}\) e quella calda cede \(Q, \Delta S_{h} = - Q / T_{h}\), segue
\[
    \Delta S = \Delta S_{c} + \Delta S_{h} = \frac{Q}{T_{c}} + \frac{-Q}{T_{h}} > 0
\]
\end{document}
\clearpage