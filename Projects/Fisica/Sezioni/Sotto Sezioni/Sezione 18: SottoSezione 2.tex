\documentclass[../../Main/Appunti Fisica.tex]{subfiles}
\begin{document}
Una \textit{macchina di Carnot} è una macchina termica operante in un solo ciclo reversibile ideale, tra due sorgenti termiche,
che fissa il limite superiore al rendimento di qualsiasi altra macchina termica.
\\ \\
A queste è legato il teorema di Carnot, di seguito riportato.
\begin{Theorem*}[di Carnot]
    Nessuna macchina termica operante tra due sorgenti termiche, può essere più efficiente di una macchina di Carnot operante tra le stesse sorgenti.
\end{Theorem*}
\begin{Proof*}
    Si suppongano due macchine operanti su due stesse sorgenti di calore. Si supponga la prima una macchina di Carnot con rendimento \(\epsilon_{C}\),
    la seconda una macchina termica con rendimento \(\epsilon > \epsilon_{C}\).
    \\ \\
    La macchina con rendimento \(\epsilon\) è utilizzata per far operare quella di Carnot come frigorifero.
    Per combinazione tra le due non vi è scambio di lavoro con l'esterno.
    \\ \\
    Assumendo come fatto che la macchina sia più efficiente del frigorifero, segue che vi è un trasferimento di energia senza aver compiuto lavoro.
    Ma ciò viola il secondo principio, segue che \(\epsilon > \epsilon_{C}\) è impossibile.
\end{Proof*}
\clearpage

\subsubsection{Ciclo di Carnot.}
Si consideri la figura di seguito riportata, rappresentante ciclo di un gas operante tra le temperature \(T_{h}\) e \(T_{c}\).
\begin{figure}[!h]
    \centering
    \begin{tikzpicture}[scale = 1, every node/.style={scale=1}]
        % Axis
        \draw [-stealth, very thick] (0, 0) -- (0, 4);
        \node [anchor = east] at (0, 5) {\(\vb{P}\)};

        \draw [-stealth, very thick] (0, 0) -- (4, 0);
        \node [anchor = north west] at (4, 0) {\(V\)};

        % Carnot period
        \node (a) [circ, color = MidnightBlue] at (1, 3) {A};
        \node (b) [circ, color = MidnightBlue] at (3, 2) {B};
        \node (c) [circ, color = MidnightBlue] at (3.5, 1) {C};
        \node (d) [circ, color = MidnightBlue] at (2, 1) {D};

        \draw [thick] (b) parabola (a);
        \draw [thick] (c) parabola (b);
        \draw [thick] (d) parabola (c);
        \draw [thick] (d) parabola (a);

        \draw [dashed] (b) -- ($ (b) + (1, 0) $);
        \node [anchor = west] at ($ (b) + (1, 0)$) {\(T_{h}\)};

        \draw [dashed] (c) -- ($ (c) + (1, 0) $);
        \node [anchor = west] at ($ (c) + (1, 0)$) {\(T_{c}\)};
    \end{tikzpicture}
    \caption{Schema del ciclo di Carnot.}
    \label{fig:11}
\end{figure}

Si hanno le seguenti trasformazioni
\begin{itemize}
    \item \(A \to B\): isoterma a temperature \(T_{h}\). Si ssorbe \(\abs{Q_{h}}\), si compie lavoro \(\vb{W_{AB}}\).
    \item \(B \to C\): adiabatica, la temperature passa da \(T_{h}\) a \(T_{c}\), si compie lavoro \(\vb{W_{BC}}\).
    \item \(C \to D\): isoterma a temperature \(T_{c}\). Si assorbe \(\abs{Q_{c}}\), si compie lavoro \(\vb{W_{CD}}\).
    \item \(D \to A\): adiabatica, la temperature passa da \(T_{c}\) a \(T_{h}\), si compie lavoro \(\vb{W_{DA}}\).
\end{itemize}
%
Il rendimento è dunque dato da
\[
    \epsilon = \vb{W_{AB}} + \vb{W_{BC}} + \vb{W_{CD}} + \vb{W_{DA}}  =  \frac{\vb{W_{mec}}}{\abs{Q_{h}}} =  1 - \frac{\abs{Q_{c}}}{\abs{Q_{h}}}
\]
Da osservazioni sperimentali \(\abs{Q_{c}} / \abs{Q_{h}} = T_{c} / T_{h}\), da cui allora
\[
    \epsilon_{C} = 1 - \frac{T_{c}}{T_{h}}
\]
\end{document}
\clearpage