\documentclass[../../Main/Appunti Fisica.tex]{subfiles}
\begin{document}
Si consideri un operaio che deve spostare una massa da un punto \textcircled{a} ad un punto \textcircled{b}, col la possibilità di scegliere tra due percorsi,
il primo più lungo ma meno ripido, il secondo l'esatto opposto.
\\ \\
Sebbene il lavoro svolto risulta uguale indipendentemente dal percorso, l'unica differenza la si ha nel tempo impiegato.
Si definisce l'energia trasferita per istante di tempo \textit{potenza}.
\[
    \mv{P} \equiv \frac{\vb{W}}{\Delta t} \implies \vb{P} = \dv{\vb{W}}{t}
\]
%
Nel SI, la potenza si misura in Watt, (\SI{}{\watt}).
\begin{center}
    \SI[per-mode = fraction, unit-color = MidnightBlue]{1}{\watt \equiv \joule \per \second}
\end{center}
\end{document}