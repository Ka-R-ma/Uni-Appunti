\documentclass[../../Main/Appunti Fisica.tex]{subfiles}
\begin{document}
\begin{Fact*}
    Un campo elettrico indotto non è conservativo.
\end{Fact*}

Si consideri una spira circolare di raggio \(r\), immersa in un campo elettrico uniforme, perpendicolare alla spira.
\\ \\
In accordo con Faraday, se il campo magnetico varia, \(\varepsilon = \dd \Phi_{B} / \dd t\), ma ciò implica la presenza di un campo \(\va{E}\) nella spira,
tangente a ogni punto della stessa.
\\ \\
Calcolando ora il lavoro necessario affinché un carica \(q\) compia un'intero giro della spira, si ha
\[
    \vb{W} = q \varepsilon = q \vb{E} 2 \pi r
\]
ma \(q \va{E}\) è la forza a cui è soggetta la carica, allora
\[
    \vb{E} = \frac{\varepsilon}{2 \pi r}
\]
%
Dunque, poiché per una spira circolare \(\Phi_{B} = \vb{B} A = \vb{B} \pi r^{2}\), segue
\[
    \vb{E} = - \frac{1}{2 \pi r} \dv{\Phi_{B}}{t} = - \frac{r}{2} \dv{\Phi_{B}}{t}
\]

Più in generale la legge di Faraday stabilisce
\[
    \oint{}{}{\va{E} \cdot}{\va{s}} = - \dv{\Phi_{B}}{t}
\]
\end{document}
\clearpage