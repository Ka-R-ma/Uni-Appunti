\documentclass[../../Main/Apputni Fisica.tex]{subfiles}
\begin{document}
Si definisce circuito RC, un circuito del tipo quello della figura di seguito riportata.
\begin{figure}[!h]
    \centering
    \begin{tikzpicture}[scale = 1, every node/.style={scale=0.75}]

        % Circuit 
        \node (source) [battery1shape, rotate = -90] at (0, 0) {};
        \node [anchor = east] at (-0.5, 0) {\(\varepsilon\)};

        \node (a) [ocirc] at ($(source.left) + (1.5, 0.5)$) {};
        \node [anchor = south] at ($(source.left) + (1.5, 0.5)$) {\(a\)};

        \node (b) [ocirc] at ($(a) + (0.25, -0.5)$) {};
        \node [anchor = east] at ($(a) + (0.25, -0.5)$) {\(b\)};

        % Cables
        \draw (source.left) to [short] ($(source.left) + (0, 0.5)$)
        ($(source.left) + (0, 0.5)$) to (a);

        \draw ($(a) + (0.5, 0)$) to [short] ($(a) + (1.5, 0)$)
        ($(a) + (1.5, 0)$) to [C] ($(a) + (1.5, -1)$)
        ($(a) + (1.5, -1)$) to [R] ($(a) + (1.5, -2)$);

        \draw (source.right) to [short] ($(source.right) + (0, -1.15)$)
        ($(source.right) + (0, -1.15)$) to ($(a) + (1.5, -2)$);

        \draw (b) -- ($(b) + (0, -1.495)$);

        \draw ($(a) + (0.125, -0.25)$) to  ($(a) + (0.5, 0)$);

    \end{tikzpicture}
    \caption{Circuito RC.}
    \label{fig:15}
\end{figure}
\clearpage

\subsubsection{Carica di un condensatore.}
Si consideri Figura \ref{fig:15}.
Dalla seconda legge di Kirchhoff, supponendo che il circuito sia chiuso con l'interruttore in \(a\), segue
\begin{equation}\label{eq:18}
    \varepsilon - IR - \frac{q}{C} = 0
\end{equation}
Supponendo che \(t = 0\) e il condensatore sia scarico, si può calcolare la corrente massima \(I_{i}\) come
\[
    I_{i} = \frac{\varepsilon}{R}
\]
%
Dall'Equazione \eqref{eq:18}, trascorso un tempo sufficientemente lungo, la corrente non circola più nel circuito, segue
\[
    Q = C \varepsilon
\]
%
Analiticamente segue
\[
    I = \dv{q}{t} = \frac{\varepsilon}{R} - \frac{q}{RC}
\]
%
Da quest'equazione si ricava una relazione per \(q\), infatti
\[
    \dv{q}{t} = \frac{\varepsilon C}{RC} - \frac{q}{RC} = - \frac{q - C \varepsilon}{RC}
\]
moltiplicando per \(dv{t}\) e dividendo per \(q - C \varepsilon\), segue integrando
\[\begin{aligned}
        \int{0}{q}{\frac{1}{q - C \varepsilon}}{q} & = - \frac{1}{RC} \int{0}{t}{}{t}                                                \\
                                                   & = \ln \left( \frac{q - C \varepsilon}{- C \varepsilon} \right) = - \frac{t}{RC}
    \end{aligned}\]
quindi
\[
    q(t) = C \varepsilon \left(1 - e^{-t/RC} \right) = Q \left(1 - e^{-t/RC} \right)
\]
da ciò, derivando, segue
\[
    I(t) = \frac{\varepsilon}{R} e^{-t/RC} = \frac{\varepsilon}{R} e^{-t/\tau}
\]
ove \(\tau = RC\).
\clearpage

\subsubsection{Scarica di un condensatore.}
Si consideri nuovamente Figura \ref{fig:15}, supponendo che il condensatore sia carico e il circuito sia chiuso in \(b\).
\\ \\
Considerando la maglia contenente condensatore e resistore, segue
\[
    -\frac{q}{C} - IR = 0
\]
ponendo \(I = \dd q / \dd t\), segue
\[\begin{gathered}
        \frac{q}{C} = - R \dv{q}{t}    \\
        \frac{\dd q}{q} = \frac{1}{RC} \dd t
    \end{gathered}\]
%
Integrando segue pertanto
\[\begin{gathered}
        \int{0}{q}{\frac{1}{q}}{q} = \frac{1}{RC} \int{0}{t}{}{t}    \\
        q(t) = Q e^{-t/RC} = Q e^{-t/\tau}
    \end{gathered}\]
da ciò derivando nuovamente, segue
\[
    I(t) = - \frac{Q}{RC} e^{-t/RC} = \frac{Q}{RC} e^{-t/\tau}
\]
\end{document}
\clearpage