\documentclass[../../Main/Appunti Fisica.tex]{subfiles}
\begin{document}
Si consideri una particella con velocità perpendicolare ad un campo uscente.
Sebbene la direzione della velocità cambi, questa rimane perpendicolare al campo. Da ciò segue che la particella segue un percorso circolare.\\
Se pertanto si applica la seconda legge di Newton, segue
\[
    \sum \vb{F} = \vb{F_{B}} = m\vb{a}
\]
ma come detto
\[
    \vb{F_{B}} = \abs{q} \vb{vB} \sin \theta = \abs{q} \vb{vB} = \frac{m \vb{v}^{2}}{r}
\]
Da quest'espressione si ricava che il raggio della traiettoria circolare, e conseguentemente la velocità angolare.
\[\begin{gathered}
        r = \frac{m\vb{v}}{\abs{q}\vb{B}} \\
        \omega = \frac{v}{r} = \frac{\abs{q} \vb{B}}{m} \\
        \tau = \frac{2 \pi r}{\vb{v}} = \frac{2 \pi}{\omega} = \frac{2 \pi m}{\abs{q} \vb{B}} \\
    \end{gathered}\]
\end{document}