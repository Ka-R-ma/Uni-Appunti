\documentclass[../../Main/Appunti Fisica.tex]{subfiles}
\begin{document}
Si definisce macchina termica un dispositivo che, assorbendo energia come calore, restituisce parte della stessa come lavoro.
\\ \\
Si consideri una macchina a vapore, qui l'acqua, sotto forma di vapore, si espande contro un pistone, producendo così lavoro.
Una volta condensatosi, il vapore si trasforma in acqua, ripetendo il ciclo.
\\ \\
Si ha così che la macchina termica ha assorbito una quantità \(\abs{Q_{h}}\) di calore, compiendo lavoro \(\vb{W}\) meccanico e cedendo calore \(\abs{Q_{c}}\).
\\ \\
Per il primo principio dunque \(\Delta E_{int} = Q + \vb{W} = Q - \vb{W_{mec} = 0}\), segue quindi
\[
    \vb{W_{TOT}} = Q_{TOT} = \abs{Q_{h}} - \abs{Q_{c}}
\]
Si definisce \textit{rendimento} di una macchina termica, il rapporto tra lavoro compiuto e calore assorbito, cioè
\[
    \epsilon \equiv \frac{\vb{W_{mec}}}{\abs{Q_{h}}} = \frac{\abs{Q_{h}} - \abs{Q_{c}}}{\abs{Q_{h}}}  = 1 - \frac{\abs{Q_{c}}}{\abs{Q_{h}}}
\]
Dalla precedente equazione segue \(\epsilon = 1 \iff \abs{Q_{c}} / \abs{Q_{h}} = 0\). Ma ciò è inconcepibile per una meccanica reale.
Da ciò segue il \textit{secondo principio della termodinamica}.
\begin{Principle*}[secondo della termica]
    L'energia non fluisce da un corpo freddo ad uno caldo, se la stessa è sotto forma di calore.    
\end{Principle*}
\end{document}
\clearpage