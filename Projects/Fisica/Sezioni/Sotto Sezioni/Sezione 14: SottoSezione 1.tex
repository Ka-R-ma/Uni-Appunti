\documentclass[../../Main/Appunti Fisica.tex]{subfiles}
\begin{document}
Si consideri Figura \ref{fig:6}.
Supponendo il piano privo di attrito, da osservazioni sperimentali si sa che se mossa da \(x_{0}\), la massa produrrà un moto oscillatorio da destra a sinistra.
\\
Dalla legge di Hooke, si ha che la forza risultante sul blocco sia
\[
    \vb{F_{k}} = -kx
\]
da cui applicando la seconda legge di Newton, segue
\[
    -kx = m\vb{x_{x}} \implies \vb{a_{x}} = -\frac{k}{m} x
\]

\subsubsection{Modello: punto materiale in moto armonico.}
Volendo dare una descrizione matematica del moto, si osserva che la massa è soggetta ad una forza \(\vb{F_{k}}\), poiché \(\vb{a_{x}} = \dv*{v}{t}\) segue dunque
\[
    \dv[2]{x}{t} = -\frac{k}{m}x
\]
da cui ponendo \(\tfrac{k}{m} = \omega^{2}\), segue
\begin{equation}\label{eq:4}
    \dv[2]{x}{t} = -\omega^{2} x
\end{equation}
Soluzione all'equazione \eqref{eq:4}, risulta essere una funzione \(x(t)\) come di seguito descritta.
\[
    x(t) = A \cos(\omega t + \phi)
\]
ove \(A\) è detta ampiezza, \(\omega\) è nota come pulsazione e \(\phi\) è chiamata fase.
\\ \\
Si osserva infatti
\begin{equation}\label{eq:5}
    \dv{x}{t} = \va{v} = -\omega A \sin(\omega t + \phi)
\end{equation}
\begin{equation}\label{eq:6}
    \dv[2]{x}{t} = \va{a} = -\omega^{2} A \cos(\omega t + \phi)
\end{equation}
%
Infine se si indica con \(\tau\) il periodo di oscillazione, e con \(f\) la frequenza, segue
\[\begin{gathered}
    \tau = \frac{2\pi}{\omega} = 2\pi \sqrt{\frac{m}{k}} \\
    f = \frac{1}{\tau} = \frac{\omega}{2\pi} \\
\end{gathered}\]
\end{document}