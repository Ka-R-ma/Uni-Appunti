\documentclass[../../Main/Appunti Fisica.tex]{subfiles}
\begin{document}
    Sia \(q\) la carica presente in un condensatore in un qualsiasi momento durante la carica.
    Dunque \(\Delta = q / C\), inoltre, il lavoro per spostare una carica \(\dd q\) è
    \[\begin{gathered}
        \dd \vb{W} = \Delta V \dd q = \frac{q}{C} \dd q \\
        \vb{W} = \int{0}{Q}{\frac{q}{C}}{q} = \frac{1}{C} \int{0}{Q}{q}{q} = \frac{Q^{2}}{2C} \\
    \end{gathered}\]
    %
    Da ciò segue pertanto
    \[\begin{aligned}
        U = \frac{Q^{2}}{2 C} = \frac{1}{2} Q \Delta V &= \frac{1}{2} C (\Delta V)^{2} \\
        &= \frac{1}{2} \frac{\varepsilon_{0} A}{d} (\vb{E}^{2} d^{2}) \\
        &= \frac{1}{2} (\varepsilon_{0} A d)\vb{E}^{2} \\
        &= \frac{1}{2} \mu d^{2}
    \end{aligned}\]
    ove \(\mu = \tfrac{1}{2} \vb{E}^{2} \varepsilon_{0}\) è detta \textit{densità di energia}.
\end{document}
\clearpage