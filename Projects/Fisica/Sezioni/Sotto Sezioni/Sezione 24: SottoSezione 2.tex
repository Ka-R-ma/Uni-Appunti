\documentclass[../../Main/Appunti Fisica.tex]{subfiles}
\begin{document}
Dati due o più resistori, questi possono tra loro essere collegati
\begin{enumerate}
    \item in serie,
    \item in parallelo.
\end{enumerate}

\subsubsection{Resistori in serie.}
Dati due resistori, questi si dicono in serie se la corrente che attraversa la prima, attraversa anche la seconda.
Cioè vale
\[
    I = I_{1} = I_{2}
\]
Ma la differenza di potenziale \(\Delta V\) è
\[\begin{aligned}
        \Delta V & = I_{1}R_{1} + I_{2}R_{2}
                 & = IR_{eq}
    \end{aligned}\]
da ciò
\[
    R_{eq} = R_{1} + R_{2}
\]

\subsubsection{Resistori in parallelo.}
Dati due resistori, questi si dicono in parallelo se sottoposte alla stessa tensione \(\Delta V\).
Vale cioè
\[
    \Delta V = \Delta V_{1} = \Delta V_{2}
\]
ma dunque
\[
    I = I_{1} + I_{2}
\]
ma si è detto che
\[
    I = \frac{\Delta V}{R_{eq}}
\]
da ciò
\[
    I = \frac{\Delta V_{1}}{R_{1}} + \frac{\Delta V_{2}}{R_{2}}
\]
pertanto, poiché \(\Delta V = \Delta V_{1} = \Delta V_{2}\), segue
\[
    \frac{1}{R_{eq}} = \frac{1}{R_{1}} + \frac{1}{R_{2}}
\]
\end{document}