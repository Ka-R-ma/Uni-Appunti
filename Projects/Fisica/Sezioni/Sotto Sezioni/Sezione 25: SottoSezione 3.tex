\documentclass[../../Main/Appunti Fisica.tex]{subfiles}
\begin{document}
Si consideri la figura di seguito riportata.
\begin{figure}[!h]
    \centering
    \begin{tikzpicture}[scale = 1, every node/.style={scale=1}]

        % Magnetic field
        \node (mField) [anchor = east] at (0, 0) {\(\va{B}\)};
        \foreach \i in {-2, -1, ..., 2}{
                \draw [-stealth, very thick] (0, 0.5 * \i) -- (1, 0.5 * \i);
            }

        % Coil & indicators
        \draw [very thick] (1.5, 1.5) rectangle (3, -1.5);

        \draw [|-|, thick] (3.5, 1.5) -- (3.5, -1.5);
        \node [anchor = west] at (3.5, 0) {\(a\)};

        \draw [|-|, thick] (1.5, -2) -- (3, -2);
        \node [anchor = north] at (2.25, -2) {\(b\)};

        \draw [stealth-] (1.5, 2) -- (2.5, 2);
        \node [anchor = south] at (2, 2) {\(I\)};

        \node [] at (2.25, 1.25) {\textcircled{1}};
        \node [] at (2.25, -1.25) {\textcircled{3}};
        \node [] at (1.75, 0) {\textcircled{2}};
        \node [] at (2.75, 0) {\textcircled{4}};

        \node [circ, color = MidnightBlue] at (2.25, 1.5) {};
        \node [circ, color = MidnightBlue] at (2.25, -1.5) {};
    \end{tikzpicture}
    \caption{Schema di una spira soggetta a campo magnetico.}
    \label{fig:16}
\end{figure}

Si osserva che sui fili \textcircled{1}, \textcircled{3} \(\va{L} \cp \va{B} = 0\), mentre per \textcircled{2}, \textcircled{4} si ha
\[
    \vb{F_{2}} = \vb{F_{4}} = Ia\vb{B}
\]
Supponendo che la spira sia incerneriata su \textcircled{3}, considerando che \(\vb{F_{2}}\) è opposta in direzione e verso ad \(\vb{F_{4}}\),
la spira inizierà a ruotare con un momento
\[\begin{aligned}
        \tau_{MAX} & = \vb{F_{2}} \frac{b}{2} + \vb{F_{4}} \frac{b}{2} \\
                   & = 2Ia\vb{B} \left( \frac{b}{2} \right)            \\
                   & = Iab\vb{B}                                       \\
                   & = IA\vb{B}                                        \\
    \end{aligned}\]
Se \(\vb{B}\) non è parallelo
\[
    \tau = IA\vb{B} \sin \theta
\]
ove \(\theta\) è l'angolo compreso tra la spira ed il campo.
\end{document}
\clearpage