\documentclass[../../Main/Appunti Fisica.tex]{subfiles}
\begin{document}
Si consideri una carica \(q\) posta al centro di una sfera di raggio \(r\).
In ogni punto della sfera \(\va{E}\) è parallelo a \(\Delta \va{A_{i}}\), rappresentante l'elemento di area \(\Delta A_{i}\), segue
\[
    \va{E} \cdot \Delta \va{A_{i}} = \vb{E} \Delta A_{i}
\]
%
Dall'Equazione \eqref{eq:14}, segue
\[
    \Phi_{E} = \oint{}{}{\vb{E}}{A} = \vb{E} \oint{}{}{}{A}
\]
ma poiché sferica, \(\dd A = A = 4 \pi r^{2}\), segue
\[
    \Phi_{E} = K_{C} \frac{q}{r^{2}} 4 \pi r^{2} = 4 \pi K_{C} q = \frac{q}{\varepsilon_{0}}
\]

Quanto detto viene generalizzato nella \textit{legge di Gauss}, con la quale si stabilisce che
\[
    \Phi_{E} = \oint{}{}{\va{E} \cdot}{\va{A}} = \frac{q_{int}}{\varepsilon_{0}}
\]
ove \(q_{int}\) è la carica presente unicamente all'interno della superficie.
\end{document}