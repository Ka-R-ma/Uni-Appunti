\documentclass[../../Main/Appunti Fisica.tex]{subfiles}
\begin{document}
Si consideri un gas che compia una trasformazione adiabatica infinitesima, con variazione \(\dd V\) di volume e una variazione \(\dd T\) di temperatura.
Segue \(\vb{W} = \vb{P} \dd V\).\\
Poiché unicamente dipendente dalla temperatura, si ha che
\[
    \dd E_{int} = n C_{v} \dd T
\]
è uguale ad un processo isovolumico che avviene alle stesse temperature.
\\ \\
Per il primo principio, segue
\[
    \dd E_{int} = n C_{v} \dd T = - \vb{P} \dd V
\]
%
Differenziando l'equazione di stato per i gas perfetti, segue
\[\begin{gathered}
    \vb{P} \dd V + V \dd \vb{P} = nR\dd T \\
    %TODO: add equation
\end{gathered}\]
Ponendo \(R = C_{p} - c_{v}\), segue che dividendo per \(\vb{P}V\)
\[\begin{aligned}
    \frac{\dd V}{V} + \frac{\dd P}{P} &= - \frac{C_{p} - C_{v}}{C_{v}} \frac{\dd V}{V}
    &= \frac{\dd}{V} (1 - y)
\end{aligned}\]
ove \(y = C_{p} / C_{v}\)
\\ \\
Integrando l'ultima equazione, segue
\[
    \ln \vb{P} + y \ln V = costante
\]
il che equivale a
\[
    \vb{P}V^{y} = costante 
\]
\end{document}
\clearpage