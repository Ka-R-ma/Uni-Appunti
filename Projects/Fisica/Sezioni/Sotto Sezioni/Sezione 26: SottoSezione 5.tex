\documentclass[../../Main/Apputni Fisica.tex]{subfiles}
\begin{document}
Il flusso magnetico è definito in maniera analoga a quello elettrico.\\
Si consideri un elemento di area \(\dd A\), di una qualsiasi superficie; se \(\va{B}\) è il campo su tale elemento,
il flusso dell'elemento sarà \(\dd A \cdot \va{B}\), da cui il flusso totale
\[
    \Phi_{B} = \int{}{}{\va{B} \cdot}{\va{A}} = \vb{B}A
\]
ove \(\dd \va{A}\) è il vettore perpendicolare alla superficie di modulo \(\dd A\).
\\ \\
Se la superficie forma un'angolo \(\theta\) con il campo, segue
\[
    \Phi_{B} = \vb{B}A \cos \theta
\]
\end{document}
\clearpage