\documentclass[../../Main/Appunti Fisica.tex]{subfiles}
\begin{document}
La differenza di potenziale ai capi di una batteria rimane costante, conseguentemente anche la corrente rimane costante.
Per tale ragione si parla di corrente \textit{continua}.\\
Quando si parla di \fem ci si riferisce alla forza elettromotrice fornita da una batteria.
\\ \\
Come logico pensare, anche una \fem ha una resistenza interna, dunque l'effettiva differenza di potenziale di una batteria è
\[
    \Delta V = \varepsilon - Ir
\]
ove \(\varepsilon\) è la forza elettromotrice propria della batteria, \(r\) la resistenza interna della stessa.
Dunque considerando figura \ref{fig:14}, segue
\[\begin{gathered}
        \varepsilon = IR + Ir \\
        I = \frac{\varepsilon}{R + r}
    \end{gathered}\]
ma quindi
\[
    \vb{P} = I \varepsilon = I^{2}(R + r)
\]
\end{document}
\clearpage