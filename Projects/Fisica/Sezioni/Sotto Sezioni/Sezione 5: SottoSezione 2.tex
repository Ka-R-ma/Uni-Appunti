\documentclass[../../Main/Appunti Fisica.tex]{subfiles}
\begin{document}
Si definisce moto uniformemente accelerato, un moto che si muove con accelerazione costante.
Cioè
\[
    a = costante = \vb{a}
\]
da cui, poiché l'accelerazione è costante, segue
\[
    x(t) = x_{0} + \vb{v}t\]
ma poiché la velocità varia linearmente
\[
    \vb{v} = \frac{v_{0} + v_{f}}{2}
\]
da cui segue
\[
    \begin{aligned}
        x(t) & = x_{0} + \frac{v_{0} + v_{f}}{2}t   \\
             & = x_{0} + v_{0}t + \frac{1}{2}at^{2} \\
    \end{aligned}
\]
\clearpage
\end{document}