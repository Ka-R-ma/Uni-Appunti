\documentclass[../../Main/Appunti Fisica.tex]{subfiles}
\begin{document}
    Si consideri un conduttore con sezione d'area \(A\), percorso da corrente \(I\).
    Si definisce \(J\) \textit{densità di corrente}, cioè la corrente per unità d'area come
    \[
    J \equiv \frac{I}{A} = n q v_{d}    
    \]
    oppure
    \[
    J = \sigma \vb{E}    
    \]
    ove \(\sigma\) è detta conducibilità, la quale è una caratteristica propria dei materiali.
    Materiali di questo tipo si definiscono \textit{ohmici}, soddisfano cioè la \textit{legge di Ohm}.
    \begin{Law*}[di Ohm]
        Per molti materiali, il rapporto tra densità di carica e campo elettrico è una costante \(\sigma\), indipendente dal campo elettrico che genera la corrente.
    \end{Law*}
    Può risultare utile stabilire una relazione tra differenza di potenziale e lunghezza, se si considerano le applicazioni pratiche, considerando il campo uniforme
    \[
        \Delta V = \vb{E} l
    \]
    da cui, considerando un filo, segue
    \[
    J = \sigma \frac{\Delta V}{l}    
    \]
    analogamente poiché \(J = I / A\), segue
    \[
    \Delta V = \frac{l}{\sigma} l = \frac{l}{\sigma A} I = RI    
    \]
    ove \(R\) è detta \textit{resistenza}, la quale può essere scritta come \(R = \Delta V / I\)
    \\ \\
    Nel SI, la resistenza è misurata in Ohm, \((\ \Omega \ )\).
    \[
    1 \quad \Omega \equiv V / A    
    \]
    Inoltre ogni resistenza è caratterizzata di una \textit{resistività}, definita come
    \[
    \rho \equiv \frac{1}{\sigma}
    \]
\end{document}