\documentclass[../../Main/Appunti Fisica.tex]{subfiles}
\begin{document}
Sebbene valide per qualsiasi tipo di campo elettrico, l'Equazioni \eqref{eq:15}, \eqref{eq:16} sono semplificabili per i campi uniformi.
\\ \\
Si considerino due punti \textcircled{a} e \textcircled{b} tra loro distanti \(d = \abs{\va{s}}\), la differenza di potenziale tra loro è
\[
    \Delta V = - \int{\textcircled{a}}{\textcircled{b}}{\va{E} \cdot}{\va{s}} = - \int{\textcircled{a}}{\textcircled{b}}{\vb{E} \cos \theta}{s} = - \int{\textcircled{a}}{\textcircled{b}}{\vb{E}}{s}
\]
posto \(va{s}\) parallelo le linee di campo.
Ma \(\vb{E}\) è costante, segue allora
\[
    \Delta V = - \vb{E} \int{\textcircled{a}}{\textcircled{b}}{}{s} = -\vb{E} d
\]
da cui segue
\[
    \Delta U = q_{0} \Delta V = - q_{0} \vb{E}
\]

Più in generale per uno spostamento \(\va{s}\), non parallelo alle linee di campo tra due punti qualunque
\[\begin{gathered}
    \Delta V = - \int{\textcircled{a}}{\textcircled{b}}{\va{E} \cdot}{\va{s}} = - \va{E} \int{\textcircled{a}}{\textcircled{b}}{}{\va{s}} = - \va{E} \cdot \va{s} \\
    \Delta U = q_{0} \Delta V = - q_{0} \va{E} \cdot \va{s} \\
\end{gathered}\]
\end{document}
\clearpage