\documentclass[../../Main/Appunti Fisica.tex]{subfiles}
\begin{document}
Si consideri una carica \(q_{0}\) immersa in un campo \(\va{E}\). Si è gia detto che la carica sarà sottoposta ad un forza \(q_{0} \va{E}\).\\
Si supponga che la carica compia uno spostamento \(\dd \va{s}\), ne segue \(\vb{W} = q_{0} \va{E} \cdot \dd \va{s}\).
Ciò implica pertanto una variazione \(\dd U = - \vb{W}\) dell'energia potenziale del sistema.
\\ \\
Considerando dunque che la carica \(q_{0}\) compia uno spostamento da un punto \textcircled{a} ad un punto \textcircled{b}, segue
\begin{equation}\label{eq:15}
    \Delta U = - q_{0} \int{\textcircled{a}}{\textcircled{b}}{\va{E} \cdot}{\va{s}}
\end{equation}
da cui, se si divide per \(q_{0}\), si ottiene la differenza di potenziale unicamente dipendente dalle cariche sorgenti.
Si definisce questi \textit{differenza di potenziale elettrico}.
\begin{equation}\label{eq:16}
    \Delta V \equiv \frac{\Delta U}{q_{0}} =  - \int{\textcircled{a}}{\textcircled{b}}{\va{E} \cdot}{\va{s}}
\end{equation}
da cui logicamente
\[
    V = \frac{U}{q_{0}}
\]
Nel SI, il potenziale elettrico è misurato in volt, (\SI[unit-color = MidnightBlue]{}{\volt}).
\begin{center}
    \SI[unit-color = MidnightBlue, per-mode = fraction]{1}{\volt \equiv \joule \per \coulomb}
\end{center}
\end{document}
\clearpage