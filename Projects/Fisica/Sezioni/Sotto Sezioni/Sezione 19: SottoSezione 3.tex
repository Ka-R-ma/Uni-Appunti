\documentclass[../../Main/Appunti Fisica.tex]{subfiles}
\begin{document}
Il primo a misurare l'intensità delle cariche elettriche fà \textit{Charles Coulomb}.
Questi dedusse, da osservazioni sperimentali, che l'intensità della forza elettrica tra due cariche puntiformi è data dalla seguente relazione.
\[
    \vb{F_{e}} = \frac{K_{C} \abs{q_{1}} \abs{q_{2}}}{r^{2}} \quad \text{legge di Coulomb}
\]
ove \(K_{C}\) è nota come costante di Coulomb pari a \SI[unit-color = MidnightBlue, per-mode = fraction]{8.99 e9}{\newton \cdot \metre \tothe{2} \per \coulomb \tothe{2}},
\(\abs{q_{1}}, \abs{q_{2}}\) sono i valori assolute delle cariche in analisi, \(r\) la distanza tra le due.
\\ \\
Generalmente conviene scrivere la costante di coulomb come
\[
    K_{C} = \frac{1}{4\pi\varepsilon_{0}}
\]
ove \(\varepsilon_{0}\) è la costante di dielettrica del vuoto, pari a \(1 / \mu_{0} c^{2}\), ove \(\mu_{0}\) è la costante magnetica del vuoto, \(c\) la velocità della luce.
\\ \\
Più in generale la forza esercitata da una carica su un'altra, è data come seguente
\begin{equation}\label{eq:13}
    \va{F}_{1, 2} = - \va{F}_{2, 1} = \frac{K_{C} \abs{q_{1}} \abs{q_{2}}}{r^{2}} \vu{r}_{1, 2}
\end{equation}
ove \(\vu{r}_{1, 2}\) è il versore da \(q_{1} \text{a} q_{2}\).
\end{document}