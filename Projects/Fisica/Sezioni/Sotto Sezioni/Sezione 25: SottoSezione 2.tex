\documentclass[../../Main/Apputni Fisica.tex]{subfiles}
\begin{document}
Si consideri un filo di lunghezza \(L\) e sezione d'area \(A\), immerso in un campo magnetico esterno \(\va{B}\). \\
Segue che la forza magnetica \(\va{F_{B}}\) sarà
\[
    \va{F_{B}} = (\abs{q} \va{v_{d}} \cp \va{B}) nAL
\]
ove \(n\) è il numero di portatori di corrente per unita di volume.
Ma \(I = n q \vb{v_{d} A}\), segue allora
\[
    \va{F_{B}} = I\va{L} \cp \va{B}
\]
ove \(\va{L}\) è il versore di \(I\) con modulo \(L\).
\\ \\
Da quanto detto, considerando ora un filo di forma arbitraria, per uno spostamento \(\dd \va{s}\), la forza magnetica sarà
\[
    \dd \va{F_{B}} = I \dd \va{s} \cp \va{B}
\]
da cui
\[
    \vb{F_{B}} = \oldInt\limits_{a}^{b}{\dd \va{s} \cp \va{B}}
\]
\end{document}
\clearpage