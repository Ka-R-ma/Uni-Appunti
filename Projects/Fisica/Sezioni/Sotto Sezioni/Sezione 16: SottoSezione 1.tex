\documentclass[../../Main/Appunti Fisica.tex]{subfiles}
\begin{document}
Si definisce \textit{calore}, lo scambio di energia attraverso la superficie che racchiude un sistema,
dovuto ad una differenza di temperatura tra sistema e ambiente esterno.

\subsubsection{Unità di calore.}
Le prime nozioni sul calore si basavano su un fluido, il \textit{calorico}.
Questi fluiva da una sostanza all'altra e ne causava una variazione di temperatura.
\\ \\
Dal nome di tale fluido fù introdotta la \textit{caloria}.
Questa era definita come quantità di energia necessaria per incrementare la temperatura di un grammo d'acqua da 14.5 a 15.5 gradi.

\subsubsection{Equivalente meccanico del calore.}
Si consideri un contenitore cilindrico riempito di acqua, contente una turbina messa in moto da due blocchi.
\\ \\
Da osservazioni sperimentali si osserva che, se trascurata l'energia persa sui supporti, l'abbassarsi dei blocchi di un tratto \(h\),
causa una perdita di energia pari a \(2gmh\)
\\ \\
Variando di poco le condizioni dell'esperimento, si osserva che
\[
    2gmh \propto m_{a} \Delta T
\]
ove \(m_{a}\) è la massa dell'acqua, e \(\Delta T\) è l'incremento di temperatura.\\
\begin{center}
    \SI[unit-color = MidnightBlue]{1}{\calorie} \(\equiv\) \SI[unit-color = MidnightBlue]{4.18}{\joule}
\end{center}
\end{document}
\clearpage