\documentclass[../../Main/Appunti Fisica.tex]{subfiles}
\begin{document}
Nella presente sezione, si riportato due modelli per la risoluzione di problemi risolvibili applicando le leggi di Newton.

\subsubsection{Modello: punto materiale in equilibrio.}
Si consideri un corpo in equilibrio sorretto da una fune, il cui schema di corpo libero è riportato nella figura di seguito.

\begin{figure}[!h]
    \centering
    \begin{tikzpicture}[scale = 1, every node/.style={scale=0.75}]

        % The mass
        \node [scale = 2, circ, color = MidnightBlue] at (0, 0) {};
        \node [anchor = east] (0, -0.125) {\(m\)};

        % The forces
        \draw[-stealth, color = MidnightBlue, thick] (0, 0) -- (0, 1);
        \node [anchor = south east] at (0, 1) {\(\vb{T}\)};

        \draw[-stealth, color = MidnightBlue, thick] (0, 0) -- (0, -1);
        \node [anchor = north east] at (0, -1) {\(\vb{F_{p}}\)};

    \end{tikzpicture}
    \caption{}
    \label{fig:4}
\end{figure}

Dalla \textit{Figura \ref{fig:4}} si osserva che:
\begin{itemize}
    \item sull'asse \textit{x} non vi sono forze, quindi
          \[
              \sum \vb{F_{x}} = 0
          \]

    \item sull'asse \textit{y} le forze agenti sono la forza peso e la tensione, quindi
          \[
              \sum \vb{F_{y}} = \vb{F_{p}} + \vb{T} = 0 \implies \vb{T} = - \vb{F_{p}}
          \]
\end{itemize}
\clearpage

\subsubsection{Modello: punto materiale sotto l'azione di forze esterne.}
Si consideri un corpo, posto su una superficie orizzontale, soggetto ad una forza verso destra, il cui schema di corpo libero è riportato nella figura di seguito.
\begin{figure}[!h]
    \centering
    \begin{tikzpicture}[scale = 1, every node/.style={scale=0.75}]

        % The mass
        \node [scale = 2, circ, color = MidnightBlue] at (0, 0) {};
        \node [anchor = east] (0, -0.125) {\(m\)};

        % The forces
        \draw[-stealth, color = MidnightBlue, thick] (0, 0) -- (0, 1);
        \node [anchor = south east] at (0, 1) {\(\vb{N}\)};

        \draw[-stealth, color = MidnightBlue, thick] (0, 0) -- (0, -1);
        \node [anchor = north east] at (0, -1) {\(\vb{F_{p}}\)};

        \draw[-stealth, color = MidnightBlue, thick] (0, 0) -- (1, 0);
        \node [anchor = south] at (1, 0) {\(\vb{F_{app}}\)};

    \end{tikzpicture}
    \caption{}
    \label{fig:5}
\end{figure}

Supponendo di voler calcolare
\begin{enumerate}
    \item l'accelerazione del corpo;
    \item la forza normale esercitata dalla superficie.
\end{enumerate}
si ha che
\begin{enumerate}
    \item lungo l'asse \textit{x}, l'unica forza è quella applicata, quindi
          \[
              \sum \vb{F_{x}} = \vb{F_{app}} = m\vb{a} \implies \vb{a} = \frac{\vb{F_{app}}}{m}
          \]

    \item lungo l'asse \textit{y}, le forze applicate sono la forza peso e la normale, poiché verticalmente non c'è accelerazione, segue
          \[
              \sum \vb{F_{y}} = \vb{N} + \vb{F_{p}} = 0 \implies \vb{N} = -\vb{F_{p}}
          \]
\end{enumerate}

\end{document}