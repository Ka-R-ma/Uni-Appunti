\documentclass[../../Main/Appunti Fisica.tex]{subfiles}
\begin{document}
Si consideri il circuito della figura di seguito riportata.
\begin{figure}[!h]
    \centering
    \begin{tikzpicture}[scale = 1, every node/.style={scale=0.75}]

        % Circuit
        \node (source) [battery1shape, rotate = -90] at (0, 0) {};

        \draw (source.left) to [short] ($(source.left) + (0, 0.5)$)
        ($(source.left) + (0, 0.5)$) to [short] ($(source.left) + (2, 0.5)$)
        ($(source.left) + (2, 0.5)$) to [R] ($(source.right) + (2, -0.5)$)
        ($(source.right) + (2, -0.5)$) to [short] ($(source.right) + (0, -0.5)$)
        ($(source.right) + (0, -0.5)$) to [short] (source.right);

        % Points of interest
        \node (a) [circ, color = MidnightBlue] at ($(source.right) + (0, -0.25)$) {};
        \node [anchor = east] at ($(source.right) + (0, -0.25)$) {\(\textcircled{a}\)};

        \node (b) [circ, color = MidnightBlue] at ($(source.left) + (0, 0.25)$) {};
        \node [anchor = east] at ($(source.left) + (0, 0.25)$) {\(\textcircled{b}\)};

        \node (c) [circ, color = MidnightBlue] at ($(source.right) + (2, -0.375)$) {};
        \node [anchor = west] at ($(source.right) + (2, -0.375)$) {\(\textcircled{c}\)};

        \node (d) [circ, color = MidnightBlue] at ($(source.left) + (2, 0.375)$) {};
        \node [anchor = west] at ($(source.left) + (2, 0.375)$) {\(\textcircled{d}\)};

        % Current
        \node (I) [anchor = south] at (1, 0.625) {\(\va{I}\)};
    \end{tikzpicture}
    \caption{Circuito con resistenza.}
    \label{fig:14}
\end{figure}

Quando una quantità di corrente \(Q\), passa da \textcircled{a} a \textcircled{b}, si ha una variazione \(\Delta U = Q \Delta V\),
la quale viene successivamente persa nel passaggio da \textcircled{c} a \textcircled{d}.
\[
    \dv{U}{t} = \dv{}{t}(Q \Delta V) = \dv{Q}{t}\Delta V  = I \Delta V
\]
da cui
\[
    \vb{P} = I \Delta V
\]
ma \(\Delta V = I R\), allora
\[
    \vb{P} = I^{2}R
\]
\end{document}