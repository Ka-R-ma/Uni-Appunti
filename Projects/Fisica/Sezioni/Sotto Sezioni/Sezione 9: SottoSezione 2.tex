\documentclass[../../Main/Appunti Fisica.tex]{subfiles}
\begin{document}

Si è precedentemente descritta la forza di attrito dinamico esercitata su un corpo in movimento su una superficie, trascurando però qualsiasi iterazione tra i due.
\\ \\
Se si considerano questi effetti si avrà che, la superficie o il mezzo attraversato producono una forza frenante \(\va{R}\).
L'intensità \(\va{R}\) dipende dalla velocità del corpo, mentre la direzione sarà sempre opposta al moto.

\begin{Remark*}
    Poiché l'intensità di \(\va{R}\) può variare in modo, complesso si considerino i modelli dei paragrafi successivi.
\end{Remark*}

\subsubsection{Modello: Forza frenante proporzionale alla velocità del corpo.}
Si consideri un corpo immerso in un liquido, o in un gas, da osservazioni sperimentali, segue
\[
    \vb{R} = -b\vb{v}
\]
ove \(b\) è una costante dipendente la mezzo attraversato.
\begin{Example*}
    Una sfera di massa \(m\) è lasciata cadere in un liquido. Supponendo che sulla sfera agiscano unicamente la forza \(\va{R} \text{e } \va{F_{g}}\), se ne studi il moto.
    \\ \\
    Dall'applicazione della seconda legge di Newton
    \[
        \sum \vb{F} = m\vb{g} -b\vb{} = m\vb{a} = m\dv{v}{t}
    \]
    risolvendo per \(\dd v/\dd t\), segue
    \[
        \dv{v}{t} = \vb{g} - \frac{b}{m}\vb{v}
    \]
\end{Example*}

\subsubsection{Modello: Forza frenante proporzionale al quadrato della velocità.}
Si consideri un corpo che si muove ad alta velocità, da osservazioni sperimentali segue
\[
    \vb{R} = DA\vb{v}^{2}\rho
\]
ove \(D\) è una costante empirica, \(\rho\) è la densità dell'aria, \(A\) l'area della sezione trasversale perpendicolare al moto.

\begin{Example*}
    Un corpo di massa \(m\) è lasciato cadere libero da uno stato di riposo.
    segue
    \[
        \sum \vb{F} = m\vb{g} + \frac{1}{2}DA\vb{v}^{2}\rho = m\vb{a}
    \]
    da cui il modello dell'accelerazione è
    \[
        \vb{a} = \vb{g} - \left(\frac{DA\rho}{2m}\vb{v}^{2}\right)
    \]
\end{Example*}
\clearpage
\end{document}