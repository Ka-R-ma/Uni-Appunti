\documentclass[../Main/Appunti Fisica.tex]{subfiles}
\begin{document}
Per poter descrivere i fenomeni naturali, è necessario misurare gli aspetti che caratterizzano gli stessi.
A ciascuna di queste misure è assegnata una grandezza fisica, di cui a seguito si riportano quelle ``fondamentali''.

\begin{itemize}
    \item lunghezza;
    \item massa;
    \item tempo.
\end{itemize}
Il dover comunicare i risultati di un esperimento, comporta la necessità di un'unità di misura univoca.
Proprio per far fronte a tale bisogno nel 1960, nacque il ``Sistema Internazionale'' (SI), il quale si occupò di stabilire le unità di misura per le grandezze fisiche.
\\
Alcune delle principali furono

\begin{itemize}
    \item \underline{il kilogrammo} (\(\SI{}{\kilogram}\)) per la massa;
    \item \underline{il metro} (\(\SI{}{\metre}\)) per la lunghezza;
    \item \underline{il secondo} (\(\SI{}{\second}\)) per il tempo;
\end{itemize}
Altre unità di misura furono stabilite, ma si parlerà di ciascuna quando necessarie.
\end{document}
\clearpage