\documentclass[../Main/Appunti Fisica.tex]{subfiles}
\begin{document}
La presente analisi del moto del proiettile è svolta sulla base di due ipotesi quali
\begin{enumerate}
    \item in caduta libera l'accelerazione rimane costante per tutto il moto;
    \item la resistenza dell'aria è trascurata.
\end{enumerate}
%
Si consideri la figura di seguito riportata.
\subfile{../Sezioni/Tikz/Figura 1.tex}

Siano \textcircled{a}, \textcircled{b}, \textcircled{c} rispettivamente: il punto di partenza del proiettile, il punto in cui il proiettile raggiunge l'altezza massima,
il punto in cui il proiettile raggiunge lo stesso livello orizzontale di partenza.
\\ \\
Dalla \textit{Figura \ref{fig:1}} si nota che:
\begin{itemize}
    \item lungo \textit{x}, la posizione del corpo varia di moto rettilineo uniforme, poiché la velocità \(\vb{v_{x}}\) rimane costante per tutto il moto.
          Dunque se si considera \(\vb{r_{x}}\) lo spostamento lungo \textit{x}, segue
          \[
              \vb{r_{x}}(t) = \vb{r_{0}} + \vb{v}t
          \]

    \item lungo \textit{y}, la posizione del corpo varia di moto uniformemente accelerato, poiché la velocità \(\vb{v_{y}}\) varia, ma rimane costante l'accelerazione dovuta alla gravità.
          Dunque se si considera \(\vb{r_{y}}\) lo spostamento lungo \textit{y}, segue
          \[
              \vb{r_{y}}(t) = \vb{r_{0}} + \vb{v_{0}}t + \frac{1}{2}\vb{a}t^{2}
          \]
\end{itemize}
\clearpage

\subsection{Gittata e altezza massima.}
\subfile{../Sezioni/Sotto Sezioni/Sezione 6: SottoSezione 1.tex}
\clearpage
\end{document}