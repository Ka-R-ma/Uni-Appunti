\documentclass[../Main/Appunti Fisica.tex]{subfiles}
\begin{document}
Oltre al campo elettrico, una carica in moto si caratterizza di un campo magnetico. \\
Si può definire questi, come la forza che il campo esercita su una particella a velocità \(\va{v}\), cioè
\[
    \va{F_{B}} = q \va{v} \cp \va{B}
\]
L'intensità di \(\va{F_{B}}\) è data come
\[
    \vb{F_{B}} = \abs{q} \vb{vB} \sin \theta
\]
ove \(\theta\) è l'angolo minore tra \(\va{v} \text{e} \va{B}\).
\\ \\
Nel SI, il campo magnetico è misurato in tesla, (\SI[unit-color = MidnightBlue]{}{\tesla}).
\begin{center}
    \SI[unit-color = MidnightBlue, per-mode = fraction]{1}{\tesla \equiv \newton \per \ampere \cdot \metre}
\end{center}

\subsection{Moto di particella carica in un campo magnetico uniforme.}
\subfile{../Sezioni/Sotto Sezioni/Sezione 25: SottoSezione 1.tex}

\subsection{Forza magnetica agente su un conduttore percorso da corrente.}
\subfile{../Sezioni/Sotto Sezioni/Sezione 25: SottoSezione 2.tex}

\subsection{Momento meccanico di una spira percorsa da corrente.}
\subfile{../Sezioni/Sotto Sezioni/Sezione 25: SottoSezione 3.tex}
\end{document}